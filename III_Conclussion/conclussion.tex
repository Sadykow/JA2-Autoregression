This paper intends to improve the state of charge prediction inside an Electronic Vehicle.
The idea is to rely only on the sensory data and a single Neural Network-based battery model.
The experiment has proven that State of Charge prediction models tends to put significant weight into the Voltage to describe the charging process during the model training process.
Adding the charge as a feature has proven that the history of the charge had a higher weight on a system than sensory data.
As a result, it leads to a decrease in accuracy since the model anticipated to have an absolute perfect history of the charge from a table.
Every next prediction, which had a small percentage of miss accuracy, only accumulated the error in every following charge estimation, leading to even worse results than pure sensory data usage.

%
%
Instead of adjusting each input feature weights manually, the best approach towards minimising the effect of the charge into an overall system is to use the autoregression training technique.
Since the sigmoid function restrained the output, the first outputs from the model at the start of the training were random values within 50\% of the expected charge.
Instead of always having perfect values, the auto-regression technique allowed models weight adjustment based on their initial error.
As a result, the optimiser performed the model fit based on the array of model outputs over evenly spaced time samples.
\textit{Despite that the modification introduced an $n_{th}^{2}$ degree computation based on the number of outputting steps, the actual prediction was performed only based on the latest sample.}
Therefore, the autoregressive model introduced no performance losses during actual utilisation. 

\begin{itemize}
    \item \textit{The hypothesis has been proven on untrained driving profiles. In some scenarios, the charge had an offset since the mismatches with the actual charge.}
    However, the prediction of the charging process or a moment of complete battery discharge always remained correct, which indicated an excellent verification method. \\
    
    \item This method is a suitable replacement for the earlier discussed NN model based only on sensory data.
    The autoregressive model implementation makes it either a solid replacement for any other SoC estimation methods like Kalman filter or an excellent addition to validating results from multiple other methods.

    \item \textbf{FUDS dataset set has a good capture of unpredicted driving behavior. Even in the scenarios which does not have a full discharge and charge process, such model has a potential to be a good profile for any further ...}

\end{itemize}