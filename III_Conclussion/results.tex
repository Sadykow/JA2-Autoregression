\subsection{Accuracy comparison}
    The newly proposed training technique was compared against a similar implementation without a modified training loop.
    Both models were provided with ideal initial results, including the State of Charge.
    
    Every further prediction replaced the actual value of charge and used as an input in the following input set, along with actual Voltage, Current and Temperature.
    Several plots on Figure~\ref{fig:init_time} outlines four stages of the testing scenarios, with different known initial charge sequences.
    \begin{figure}[htbp]
        \centering
        % DST based tests
        \begin{subfigure}[b]{0.475\textwidth}
            \centering
            \includesvg[width=\linewidth]{III_Conclussion/im_time/train-iCharging.svg}
            \caption{Charging process}
        \end{subfigure}
        \hfill
        \begin{subfigure}[b]{0.475\textwidth}
            \centering
            \includesvg[width=\linewidth]{III_Conclussion/im_time/train-iCharged.svg}
            \caption{Charged state}
        \end{subfigure}
        \hfill
        \begin{subfigure}[b]{0.475\textwidth}
            \centering
            \includesvg[width=\linewidth]{III_Conclussion/im_time/train-iDischarging.svg}
            \caption{Discharging process}
        \end{subfigure}
        \begin{subfigure}[b]{0.475\textwidth}
            \centering
            \includesvg[width=\linewidth]{III_Conclussion/im_time/train-iDischarged.svg}
            \caption{Discharged state}
        \end{subfigure}
        \caption{Different initial periods of model validation}
        \label{fig:init_time}
    \end{figure}
    \begin{itemize}
        \item To verify the efficiency, model also was compared against the two other profiles cycling profiles. \\
        \item \textbf{model has tendency to put a lot of weight into the Voltage. This way. weights will fall into SoC instead to preserve longer dependency.} \\
        \item \textit{Different comparison, plots. Relative timing and what mechanism in implementation with lists and tensors increased it. Accuracies.} \\
        \item \textbf{Comaprison of different Windows size. Time it takes to compute that is too long. Need to overcome it, somehow.} \\
    \end{itemize}
    Plots above outline different initial starting points and show the convergence over time. \textcolor{red}{Stupid idea, that's not helpful}
\subsection{Optimal output step size}
    The optimal size of the output steps for better capturing the State of Charge was determined through experimental training.
    The following table outlines the ten training iterations of each driving profiles until it reached accuracy below 2.5\%.
    \begin{table}[htbp]
        \centering
        \caption{Percentage accuracy evaluation with increasing output step size}
        \label{tab:out_steps}
        \begin{tabular}{ p{1.5cm} || p{1.5cm} p{1.5cm} p{1.5cm} || p{1.5cm}  }
            \hline
            Steps & DST & US06 & FUDS & Epochs \\
            \hline
            10 & 2.94 & $\geq\sim 15$ & $\geq\sim 15$ & 8 \\
            15 & 0.90 & 10.67  & 11.10  & 8 \\
            20 & |    & 0.68   &  4.25  & 10 \\
            25 & |    & |      &  3.48  & 25 \\
            30 & |    & |      &  0.59  & 19 \\
            \hline
        \end{tabular}
    \end{table}
    
    %
    %
    The Dynamic Stress Test (DST) has a stable trend of current consumption and regenerative process.
    Therefore it did not take long to capture the behaviour of the charge.
    A single cycle on subfigure~\ref{subfig:res_DST} provides an example of accurate feed-forward charge estimation with output steps between 10-15 samples.
    The US06 and FUDS have the more aggressive current usage, leading to trendier voltage readings.
    As per subfigures~\ref{subfig:res_US} and~\ref{subfig:res_FUDS}, it took twice more output sample to achieve similar accuracy to DST based model. 
    Any further increase can capture even more complicated scenarios, leading to a longer training time.
    However, the amount of epoch per training model to produce the best fit decreased with every added single step.
    \begin{figure}[htbp]
        \centering
        % DST based tests
        \begin{subfigure}[b]{0.325\textwidth}
            \centering
            \includesvg[width=\linewidth]{III_Conclussion/im_best/DST-8.svg}
            \caption{Best DST based model after 8 iterations with 15 steps}
            \label{subfig:res_DST}
        \end{subfigure}
        \hfill
        \begin{subfigure}[b]{0.325\textwidth}
            \centering
            \includesvg[width=\linewidth]{III_Conclussion/im_best/US06-10.svg}
            \caption{Best US06 based model after 10 iterations with 20 steps}
            \label{subfig:res_US}
        \end{subfigure}
        \hfill
        \begin{subfigure}[b]{0.325\textwidth}
            \centering
            \includesvg[width=\linewidth]{III_Conclussion/im_best/FUDS-19.svg}
            \caption{Best FUDS based model after 19 iterations with 30 steps}
            \label{subfig:res_FUDS}
        \end{subfigure}
        \caption{Best training results over 3 different driving profiles}
        \label{fig:Models_res}
    \end{figure}