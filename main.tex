\documentclass[fleqn,10pt]{olplainarticle}
% Use option lineno for line numbers 

\title{Critical analysis of Lithium Battery State of Charge Estimation using Memory Mechanisms of Machine Learning}

\author[1]{Mr Marat (Matt) Sadykov}
\author[2]{Dr David Holmes}
\affil[1]{Queensland University of Technology}
%\affil[2]{Address of second author}

\keywords{LSTM, Auto-Regression, SoC}

\begin{abstract}
    \textbf{Implementation of reliable algorithm for determening State of Charge (SoC) of Lithium Ion (Li-Ion) battery in Electrical Vehicle (EV) is one of the major tasks in the Battery Management System (BMS).
    The article proposes a modefied version of a Long Short-Term Memory Recurrent Neural Network (LSTM-RNN) with Auto-regression mechanism.
    The application for this mechanism defined in the feed-forwarding the output of the Machine Learning (ML) model back to itself as an input.
    Initial value of State of Charge becomes another input feature, along with battery Voltage, Current and Temperature.} \\[1pc]
    \textbf{Two dataset were taken from Center for Advanced Life Cycle Engineering (CALCE) for A123 Batteries.
    They consist of Dynamic Stress Test (DST), Supplemential Federal Test Procedure-driving Schedule (US06) and Federal Urban Driving Schedule (FUDS).
    Several combitation of those test on training and validation said acted as cross-validation mechanism as well as time for model to train and number of input samples.} \\[1pc]
    \textbf{LSTM with Auto-Regression achives a root mean squared error of 0.0268 ** ** at early steps of training and improves to ** ** ** with every epoch.
    The final goal is to bring the model to a state at which it can perform an online training on Small On-Chip Computers, by using only recently measured sensor data.}
\end{abstract}

\begin{document}

\flushbottom
\maketitle
\thispagestyle{empty}
%%% Introduction
\IEEEPARstart{T}{he} rapid growth of the Electrical Vehicle (EV) market has led to the research of better energy storage.
Lithium-Ion (Li-ion) batteries offer high power density and a low self-discharge rate~\cite{han_review, en13082106}.
Such efficiency made them highly used in the EV application, on top of smartphones or other portable electronic devices.
Misusing those batteries may lead to rapid battery life degradation (ageing) and damage to a device on a small scale or environmental catastrophes on a big.
The only solution to an inefficiently used aged battery is recycling and replacement, increasing the usage cost~\cite{skeete_beyond_2020}.
The vital requirement for EVs powered with Li-ion batteries is the safe and efficient operation of the integrated Battery Management System (BMS).

%
%
State of Charge became an issue of great importance in the BMS application.
Since one of the properties of a manufactured Li-ion battery is internal resistance, the voltage reading over a cell becomes unreliable during unstable discharge.
The voltage drop leads to inaccurate voltage and temperature sensor readings during a system operation.
Since the depletion of a battery until it cannot supply enough current is undesirable for more prolonged usages, it is crucial to implement a solution to overcome such issues.
A simple battery model that accumulates the multiple of an ideal internal resistance and current consumed by a system became a commonly used solution.
Due to battery ageing over more prolonged use, the resistance diversifies due to varying battery utilised temperatures.
Eventually, such a simple battery model becomes inefficient and only leads to failure~\cite{fenrg.2019.00065}.
Working temperature, stored charge and maximum output current differed from a freshly manufactured cell.
Therefore, failure to prevent overrated usage leads to battery damage and wasting cell potential earlier than anticipated by manufacturers.
The solution is to use the State of Charge (SoC) as a battery state reference.

%
%
Several methods have been used to make an accurate estimation.
The most commonly used and simple to implement is the Coulomb Counter (CC), integral of the current over time.
However, besides the limitation of not telling the initial SoC, it cannot converge to the actual State of Charge over more prolonged usages~\cite{ng_enhanced_2009}.
The reason for that is the method itself.
The CC model does not consider variations in the battery properties due to degradation, reporting charge percentage by a scale from the first usage.
% \textcolor{red}{(Several other studies if I want to reference them).}
Different and more complicated methods approach this problem in a chemical or complex modelling way.
The methods explored in this work are based on statistical Neural Network models.

%
%
\textbf{Add references}
In recent years, there have been a few different approaches in Artificial Intelligence development for battery charge estimation: Fuzzy logic, Support Vector Machine, and Recurrent Neural Network (RNN).
The RNNs are more commonly used due to their various applications, support from different developers, and being widely tested by researchers.
There have been multiple publications around extensively used Time-Series models, such as Long short-term memory (LSTM) or Gated recurrent units (GRU), for battery state of charge prediction.
\textcolor{red}{A previously published paper of this research [Sadykov, 1] analysed multiple different implementations of RNN models to determine the State of Charge from sensory data like Voltage, Current and Temperature.}
The research results have concluded that sensory data in one driving behaviour is efficient to estimate but may fall under another.
Thus, the battery state became a matter of not the battery itself but the way it has been used.
Besides, the SoC can fail under-voltage drop under some scenarios, although not as significant and only at certain \textbf{rate conditions}.
The potential solution to the problem was integrating SoC as one of the input features to a NN model.
The limitation here is to get the State of Charge in real-time.
There are two approaches to that: Charge estimation from other means, CC or three feature-based NN methods, or the Feed-Forward method to propagate the output value of the charge as an input into the next prediction.

%
%
In this paper, the Long short-term memory model will implement a 4-feature estimator, with a history of 5 to 10 minutes usage, to predict the current State of Charge and propagate the result further.
A novel method of the training loop model will be proposed to eliminate the possibility of accumulating error along with the value of a charge.
It utilises the Autoregressive technique to introduce an adaptive and robust solution to training output inaccuracy over time.
%It is meant to force the model to consider the potential of having variations in the State of Charge history and yet not fail the prediction output further.
Furthermore, the model is forced to consider a miss accuracy of its State of Charge prediction before loss calculation.

%
%
\textbf{Replace numbers with Latex hyperlinks}
This paper is organised as follows: a methodology for an RNN model discussed in Section 2.
The details of how auto-regression has been utilised are in Section 3.
Subsections 4.1 and 4.2 separate model validation points and parameter estimation processes.
Finally, section 5 concludes the research by outlining several observations, which may require separate consideration.
% Most were isolated to closed scenarios with provided data or from battery cycling machines.
% The most promising approach to improve a model and make it more universal is to increase complexity. While some introduced deeper layer newtwork, others added additional mechanisms to already used.
\begin{landscape}
    \begin{figure}[ht]
        \centering
        \includegraphics[width=0.9\linewidth]{II_Body/images/Windowing4f-A3.jpg}
        \caption{Data Windowing scheme at 1Hz sampling rate. For visualisation purposes, the $s$-step has been used as 250 seconds, which is different from the actual implementation. The initial index $i$ was kept as a value close to the beginning of the data, around zero.}
        \label{fig:Windowing}
    \end{figure}
\end{landscape}
%%% Body
The regular training procedure for a 4-feature model has a significant limitation. The model does not take into account the Feed-Forward application of the prediction output.
In the case of excellent input values, the output is expected to be within~1 \% miss accuracy, since from 500 of ideal SoC values estimation of 501 is a trivial task, not requiring a Neural Network to perform.
\begin{figure*}[htbp]
    \centering
    % DST based tests
    \begin{subfigure}[b]{\columnwidth}
        \centering
        \includesvg[width=\linewidth]{III_Conclussion/im_compare/FUDS-val-48.svg}
        % \includesvg[width=\linewidth]{III_Conclussion/Models/Sadykov2020/FUDS-models/SMFUDS-val-9.svg}
        \caption{Regular training process snapshot}
        \label{subfig:regular_tr}
    \end{subfigure}
    \hfill
    \begin{subfigure}[b]{\columnwidth}
        \centering
        % \includesvg[width=\linewidth]{III_Conclussion/Models/Sadykov2020/FUDS-models/SMFUDS-FF-9.svg}
        \caption{Feed-Forward validation process snapshot}
        \label{subfig:regular_ts}
    \end{subfigure}
    \caption{Comparison between training and testing accuracies of a 4-featured based model with a default training and testing loop}
    \label{fig:regular_tr}
\end{figure*}
% \begin{figure}[ht]%[htbp]
%     \centering
%     \includegraphics[width=0.7\linewidth]{II_Body/images/multistep_autoregressive.png}
%     \caption{documented way}
%     \label{fig:autoregressive}
% \end{figure}

%
%
Two subplots, \mbox{Figure~\ref{fig:regular_tr}}, demonstrate the prediction results of a four feature-based trained model against a single battery cycle of DST driving.
\mbox{Subfigure~\ref{subfig:regular_tr}} demonstrates the prediction based on the always known perfect State of Charge, opposite on \mbox{subfigure~\ref{subfig:regular_ts}} with only perfect initial, and every following sample gets fed-forward.
The loss axis has been dropped from plotting due to high inaccuracy.
\ifthenelse{\boolean{thesis}}{
    The implementation of this prediction method is presented in \mbox{Appendix~\ref{app:Feed-Forward}}.
}{}
It demonstrates how the appended charge output model accumulates the error with every dependant input in a single prediction.
If that output will be used for further prediction and the model keeps preserving the dependency, the miss accuracy value rises non-linearly.
The reason for that lies in the amount of weight the model places on the State of Charge input feature.
For a better weights balance, the training procedure must be modified to consider the possibility of an inaccuracy in the input charge data.
The diagram in \mbox{Figure~\ref{subfig:testing}} illustrates regular training and testing procedures for a model to produce output.
The implementation has been based on contributions from the original framework developers ~\cite{time_2020}.
Details have been attached to Appendix~\ref{app:AutoFeedback}.
\begin{figure*}[htbp]
    \centering
    % DST based tests
    \begin{subfigure}[b]{0.85\textwidth}
        \centering
        % \includegraphics[width=\linewidth]{II_Body/images/IMG_20210524_133103.jpg}
        \includegraphics[width=\linewidth]{II_Body/images/Autoregression-Training.png}
        \caption{Custom autoregressive training procedure}
        \label{subfig:testing}
    \end{subfigure}
    \hfill
    \begin{subfigure}[b]{0.85\textwidth}
        \centering
        % \includegraphics[width=\linewidth]{II_Body/images/IMG_20210524_133052.jpg}
        \includegraphics[width=\linewidth]{II_Body/images/Autoregression-Testing.png}
        \caption{Regular testing and validation procedure}
        \label{subfig:training}
    \end{subfigure}
    \caption{Comparison between training and testing accuracy of a 4-featured based model with a default training and testing loop.}
    \label{fig:training_testing}
\end{figure*}

%
%
The training procedure for the regular LSTM model must be modified to consider potential inaccuracy using the autoregression technique.
%\textcolor{red}{I do not feel comfortable referencing the documentation, but what choice do I have now.}
The diagram in \mbox{Subfigure~\ref{subfig:training}} demonstrates the procedure for the model call using autoregression.
Unlike regular LSTM, training and testing differentiate from each other.
If the testing procedure remained unchanged, the training performs multiple calls during a single-window sample processing.
Every new call outputs the results and feeds again into the same model, with one sample from each sensor.
Each output also contained a model state, containing the values stored in the cell, preserving dependency between model calls.
State output is used only for internal model processing.
Every output of every step has been stored as an array.
With a new approach, an optimiser will compare an array of predicted samples against the true values of the SoC.
This way meant to increase the model fit process.
The more output samples model returns during the training, the better the real-time prediction against aggressive driving profiles.
\mbox{Figure~\ref{fig:modefied_tr}} contains a similar test as without autoregression.
Even though the accuracy with tabled samples has decreased, its feed-forward prediction accuracy has significantly increased.
\begin{figure*}[htbp]
    \centering
    % \begin{subfigure}[b]{\columnwidth}
    %     \centering
    %     \includesvg[width=\linewidth]{III_Conclussion/Models/Sadykov2021-30steps/FUDS-models/SMRFUDSval-19.svg}
    %     \caption{Modified training process}
    % \end{subfigure}
    % \begin{subfigure}[b]{\columnwidth}
    %     \centering
    %     \includesvg[width=\linewidth]{III_Conclussion/Models/Sadykov2021-30steps/FUDS-models/SMRFUDS-FF-19.svg}
    %     \caption{Feed-Forward validation process}
    % \end{subfigure}
    \caption{Comparison between training and testing accuracies of a 4-featured based model with a modified training and default testing loop}
    \label{fig:modefied_tr}
\end{figure*}
Time-series prediction of a neural model relies on the input history of equally distributed samples.
\ifthenelse {\boolean{thesis}}{As discussed in earlier chapters, the LSTM model is a recurrent neural network designed to solve the vanishing gradient problem by remembering (preserving) the long dependencies \cite{rasifaghihi_predictive_2020}.}
{The LSTM model is a recurrent neural network designed to solve the vanishing gradient problem by remembering (preserving) the long dependencies~\cite{rasifaghihi_predictive_2020}.}
The cells inside the model act as memory units to preserve the dependence.
Therefore, the output is closely dependent on the previous input samples.
Unlike the normal RNN, and the more modern version GRU, LSTM has a more complicated structure constructed from several logical gates~\cite{LSTM_Hochreiter1997}.
It is the most widely used type of model.
\ifthenelse {\boolean{thesis}}{Chapter~\ref{cha:Analysis} provides a summary of the LSTM cell logic, with corresponding equations explaining the gates logic in detail.} 
{\mbox{Figure~\ref{fig:LSTM-cell2}} provides a summary of the cell logic.
It utilises three gates: forget $f_t$, input $i_t$ and output $o_t$, \mbox{Equation~\ref{eq:LSTM-gates2}}.
The decisions are based around sigmoid $\sigma$ function~\ref{eq:sigmoid2}.
With default $tanh$ as activation function, \mbox{Equation~\ref{eq:LSTM-output2}} describes the procedure for cell state update and further propagation.
Output variables $h_t$ and $c_t$ represent memory cell output and the cell state at timestamp $t$.
\begin{equation}
    \sigma(x) = \frac{1}{1+e^{-x}}
    \label{eq:sigmoid2}
\end{equation}
\begin{figure}[htbp]
    \centering
    \includegraphics[width=\linewidth]{II_Body/LSTM/images/LSTM.jpg}
    \caption{Long Short-Term Memory Cell}
    \label{fig:LSTM-cell2}
\end{figure}
\begin{equation}
    \begin{split}
        f_t &= \sigma \left(W_f \left[h_{t-1}, x_t \right] + b_f \right) \\
        i_t &= \sigma \left(W_i \left[h_{t-1}, x_t \right] + b_i \right) \\
        o_t &= \sigma \left(W_o \left[h_{t-1}, x_t \right] + b_o \right) \\    
    \end{split}
    \label{eq:LSTM-gates2}
\end{equation}
\begin{equation}
    \begin{split}
        c_t &= f_t c_{t-1}+i_t \times tanh \left(W_c \left[h_{t-1}, x_t \right] + b_c \right) \\
        h_t &= o_t*tanh \left(c_t \right)
    \end{split}
    \label{eq:LSTM-output2}
\end{equation}
}

%
%
The LSTM model has been used widely in stock-price prediction or weather forecasting. 
%? add references if I feel to it
However, unlike State of Charge estimation, which commonly uses $V$, $I$, and $T$ as inputs, those methods utilise the output feature as an input to the subsequent prediction to propagate results further and calculate the time before a critical event occurrence.
Besides, methods like weather forecasting for a week are not limited by lacking output data, since the searched criteria are always known or will be known once they happen, allowing updates and improvements of follow-up predictions.
On the contrast, a battery's actual State of Charge cannot be directly determined or measured making verification against previously-made predictions difficult without additional battery modelling techniques or laboratory equipment.
%, not to mention having it integrated into an Electric Vehicle's accumulator.
It can be determined with a proper battery cycler, performing a set of pulse tests, but this is infeasible for practical applications.
%utilising a battery and affecting its charge and remaining life.
% (As opposed to the SoC estimate, where getting actual values to require a battery cycler capable ... ) 
%%%%%%%%%%%
%Unlike the charge estimation, which can only output a single value based on a history of samples, they are not limited to ... . Therefore, it does not require the output as input since the truth will become known in due time.
As such, to include the charge in the process, SoC as a learning input is used, later making a predicted array of values used on testing.
This introduces potential issues for error accumulation with every evaluation that will be addressed.

%
%
The best way to use the performance of the stateless LSTM model is through training with a data windowing technique.
The NN model will receive a fixed set of equally-distributed time samples at each time prediction.
Every next forecast will shift the time window by a constant step $s$, until all possible combinations of time slices go through the model.
That approach is referred to as a stateless model, which only sees dependencies over input samples rather than preserving every received input, like in stateful implementations.
It also allows the order of the windows to be shuffled to avoid overfitting.
Since no dropout was applied before the model's output, a set of strategies has been applied to update the learning rate and rollback before early stopping to assist the fitting process.
\ifthenelse {\boolean{thesis}}{Chapter~\ref{cha:Analysis} at subsections~\ref{subsec:l-rate} and~\ref{subsec:t_model} explain those two methods, which have already proven to be effective at training SoC models.} 
{Early research on published methods has already utilised those two methods to assist in the models' training process [Sadykov, 1].
It involved a scheduled learning rate value update with every passing epoch for as long as the accuracy kept improving with every passing iteration, as well as assisting the models' recovery in the event of overfitting by double reduction of the value either until the models' return to the same minimal optimisation or finalising the optimum result reach.}
As a result, a NN model will learn dependency between a fixed amount of equally distributed time samples $n$ and yet be independent from the order of the inputs.
\ifthenelse {\boolean{thesis}}{\mbox{Figure~\ref{fig:Windowing}} adapts the earlier Figure~\ref{fig:Windowing3f}, demonstrating how the input dataset is constructed and ordered into a 3-dimensional dataset, with four features (Voltage, Current, Temperature and added initial SoC), 500 timestamps and around a hundred thousand samples to fit on.}
{\mbox{Figure~\ref{fig:Windowing}} demonstrates how the input dataset is constructed and ordered into a 3-dimensional dataset, with four features, 500 timestamps and around a hundred thousand samples to fit on.}
Due to the size of the windows, equivalent to 8 and a quarter minutes of a discharge process, no batching mechanism has been used to reduce computational load and avoid 4-dimensional matrix management.

%
%
\ifthenelse {\boolean{thesis}}{Similar to Chapter~\ref{cha:Analysis}, the mean and standard deviation has been used to normalise all data to speed up the training process.}
{The mean and standard deviation has been used to normalise all data to speed up the training process.}
The normalisation constant from training input samples has been used for all further validation and testing sets to ensure the right trends.
The state of charge narrowed between 0 and 1 to represent the percentage charge to two decimals.
\mbox{Table~\ref{tab:params}} highlights the parameters required to define the initial model, where $s$ defines output step size, which will be justified later in Section~\ref{sec:feed}.
A $\sigma$ function as an output justifies the charge normalisation between 0 and 1.
\ifthenelse {\boolean{thesis}}{The number of neurons has been selected based on the results made in Chapter~\ref{cha:Analysis}, Section~\ref{sec:AN:Results}.
Even though the number of neurons was kept as per the performance result table, only one layer has been utilised due to manual implementation of the model and the inability to validate the multilayer implementation correctness against some published or already-used models.
It was decided to stick to known approaches to validate the efficiency of the newly-proposed technique.
\begin{table}[ht]
    \renewcommand{\arraystretch}{1.3}
    \caption{Model structure and parameters}
    \centering
    \label{tab:params}
    \begin{tabular}{ l l l }
        \hline\hline \\[-4mm]
        Input     & $shape= \left( 1,500,4 \right)$ & $batch=1 $  \\
        \hline
        LSTM      & $activation= 'tanh'$ & $units=131$  \\
        \hline
        Dropout   & $0.0$ &   \\
        \hline
        Output    & $activation= \sigma\left(s, 1 \right)$ &   \\
        \hline\hline
    \end{tabular}
\end{table}
}
{The number of neurons has been selected based on the results made based on earlier discoveries of the most optimal hyperparameters set [Sadykov, 1].
Results were adapted to the new custom-implemented model to validate the technique's efficiency in a comparable way with already-tested approaches.
\begin{table}[ht]
    \renewcommand{\arraystretch}{1.3}
    \caption{Model structure and parameters}
    \centering
    \label{tab:params}
    \resizebox{\columnwidth}{!}{
    \begin{tabular}{ l l l }
        \hline\hline \\[-4mm]
        Input     & $shape= \left( 1,500,4 \right)$ & $batch=1 $  \\
        \hline
        LSTM      & $activation= 'tanh'$ & $units=131$  \\
        \hline
        Dropout   & $0.0$ &   \\
        \hline
        Output    & $activation= \sigma\left(s, 1 \right)$ &   \\
        \hline\hline
    \end{tabular}
    }
\end{table}
}

%
%
\ifthenelse {\boolean{thesis}}
{
The optimisation algorithm for the fitting process has been selected as the regular Adam method, which was highlighted in Chapter~\ref{cha:Analysis}, \mbox{Algorithm~\ref{alg:Adam}}, with the corresponding hyperparameters, \mbox{Table~\ref{tab:uni-hyperparams}}.
The selection of this optimiser is made to allow comparison to two early-created LSTM-based models which use the same optimiser in Chapter~\ref{cha:Analysis}.
}
{
The optimisation algorithm for the fitting process has been defined by Adam, \mbox{Algorithm~\ref{alg:copyAdam}}, with the corresponding hyperparameters on \mbox{Table~\ref{tab:newM-params}}.
%\textcolor{red}{Try to use Robust Adam instead, because why the hell not since I lost one month of my life to implement that cursed algorithm from Javids miss-typed notes? Complete this section with details as per Gareth Javid's implementation if RoAdam will be able to produce a faster fitting.}
\begin{algorithm}\captionsetup{labelfont={sc,bf}, labelsep=newline}
    \caption{Adaptive Moment Estimation (Adam) optimisation}
    \begin{algorithmic}[1]
        \STATE \textbf{Number of input samples} \\ $N\gets length(\textit{input data})$\\
        \STATE \textbf{Size of windows} \\ $S\gets length(V_{i..n})$\\
        \STATE \textbf{Output steps} \\ $O\gets length(V_{i..n})$\\
        \STATE Input: $x_n = [V_{i..n}, I_{i..n}, T_{i..n}, SoC_{(i-1)..(n-1))}]$ \\
        - Shape: $X = (N, S, 4)$
        \STATE Output:$y_n = [SoC_{(n-o)..n}] - $Shape:$Y = (N, O, 1)$
        \STATE Define Loss function: $L$ \\
                Get hyperparameters: $\alpha, \beta_1, \beta_2, \epsilon$
        \WHILE{$W_t \text{ does not converge}$}
        \STATE $t \gets t+1$
        \STATE $g_t \gets \nabla_W L_t (W_{t-1})$ \COMMENT{Obtain gradient}
        \STATE $m_t \gets \beta_1 m_{t-1}+(1-\beta_1) g_t $ \COMMENT{$1_{st}$ moment calculation}
        \STATE $\upsilon_t \gets \beta_2 \upsilon_{t-1}+ \left(1-\beta_2 \right)g^2_t $ \COMMENT{$2_{nd}$ moment calculation \label{alg:Adam-Line-2Moment}}
        \STATE $\hat{m_t} \gets \frac{m_t}{1-\beta^t_1}$ \COMMENT{Corrected $\hat{m_t}$}
        \STATE $\hat{\upsilon_t} \gets \frac{\upsilon_t}{1-\beta^t_2} $ \COMMENT{Corrected $\hat{\upsilon_t}$}
        \STATE $W_t \gets W_{t-1}- \alpha \frac{\hat{m_t}}{\sqrt{\hat{\upsilon_t}}+\epsilon} $ \COMMENT{Update parameters}
        \ENDWHILE
    \end{algorithmic}
    \label{alg:copyAdam}
\end{algorithm}
\begin{table}[htbp]
    \renewcommand{\arraystretch}{1.3}
    \caption{Optimiser Hyper-Parameters}
    \centering
    \label{tab:newM-params}
    \resizebox{\columnwidth}{!}{
    \begin{tabular}{ l l l l l l }
        \hline\hline \\[-3mm]
        $\alpha$ & $\beta_1 $ & $\beta_2$ & $\beta_3$ &  $\epsilon$ \\
        \hline
        Linear         &  &  &  & \\% 0.0000001
        Scheduler      & $0.9$ & $0.999$ & $0.999$ &$10^{-8}$ \\% 0.0000001
        (0.001-0.0001) &  &  &  & \\% 0.0000001
        \hline\hline
    \end{tabular}
    }
\end{table}
}
\section{Error Correction} \label{sec:error}
\textit{I need a methematician here. We have 3 knowns samples and 4 corresponding. Rigth now it is a simple Absolute difference between previos 3 to get the 4th one. Mahsa was interested, let's hope it can be improved. Simple math here and sudo-logic on how I used to detemine the sign toward which to correct.}
%%% Conclusion
\subsection{Accuracy comparison}
    The newly proposed training technique was compared against a similar implementation without a modified training loop.
    Both models were provided with ideal initial results, including the State of Charge.
    Every further prediction replaced the actual charge value and was used as an input in the following input set, along with actual Voltage, Current and Temperature.
    
    %
    %
    Several plots in Figure~\ref{fig:diff_prof_compare} outline demonstrates the process of model validation and testing.
    The model has been trained and validated against FUDS-profile, subfigure~\ref{subfig:FUDS_diff_prof_compare}, and tested again DST on subfigure~\ref{subfig:DST_diff_prof_compare} and US06 on subfigure~\ref{subfig:US_diff_prof_compare}.
    The plots are based on the latest output from the training loop.
    The entire training process has been logged, and every model is saved as a separate checkpoint to determine the most efficient model in both validation and testing.
    Figure~\ref{fig:res_performance} demonstrates selecting the best model, marking the iteration which produced the lowest error for all three profiles.
    This way, the selection process can determine the iteration at which the model reached the lowest error across all three profiles. 

    %
    %
    The offset in the validation can be explained by the amount of weight placed into the known SoC, unlike with 3-feature models, where voltages act as the primary characteristic.
    In the State of Charge estimation, the significant impact is affected by current, and since the State of charge is the function of current and time, the weight the properly applied to the correct feature over the training process.
    Any other training process appeared to be unnecessary and may lead to overfitting.
    Figure~\ref{fig:Models_res} demonstrates the best feed-forward prediction for all three profiles.
    % \begin{figure}[htbp]
    %     \centering
    %     % DST based tests
    %     \begin{subfigure}[b]{0.475\textwidth}
    %         \centering
    %         \includesvg[width=\linewidth]{III_Conclussion/im_time/train-iCharging.svg}
    %         \caption{Charging process}
    %     \end{subfigure}
    %     \hfill
    %     \begin{subfigure}[b]{0.475\textwidth}
    %         \centering
    %         \includesvg[width=\linewidth]{III_Conclussion/im_time/train-iCharged.svg}
    %         \caption{Charged state}
    %     \end{subfigure}
    %     \hfill
    %     \begin{subfigure}[b]{0.475\textwidth}
    %         \centering
    %         \includesvg[width=\linewidth]{III_Conclussion/im_time/train-iDischarging.svg}
    %         \caption{Discharging process}
    %     \end{subfigure}
    %     \begin{subfigure}[b]{0.475\textwidth}
    %         \centering
    %         \includesvg[width=\linewidth]{III_Conclussion/im_time/train-iDischarged.svg}
    %         \caption{Discharged state}
    %     \end{subfigure}
    %     \caption{Different initial periods of model validation}
    %     \label{fig:init_time}
    % \end{figure}
    % Plots above outline different initial starting points and show the convergence over time. \textcolor{red}{Stupid idea, that's not helpful. How about I add some red line at very beginning, indicated where did I start initially and how far it plotted itself. Accumulate 500 initials steps and then plot them seperately on top to show the indication.}
    \begin{figure}[htbp]
        \centering
        \begin{subfigure}[b]{0.325\textwidth}
            \centering
            \includesvg[width=\linewidth]{III_Conclussion/Models/Sadykov2021-30steps/FUDS-models/SMRFUDS-FF-19.svg}
            \caption{FUDS trained model}
            \label{subfig:FUDS_diff_prof_compare}
        \end{subfigure}
        \hfill
        \begin{subfigure}[b]{0.325\textwidth}
            \centering
            \includesvg[width=\linewidth]{III_Conclussion/Models/Sadykov2021-30steps/FUDS-models/SMRFUDS-Test One-19.svg}
            \caption{Testing against DST profile}
            \label{subfig:DST_diff_prof_compare}
        \end{subfigure}
        \hfill
        \begin{subfigure}[b]{0.325\textwidth}
            \centering
            \includesvg[width=\linewidth]{III_Conclussion/Models/Sadykov2021-30steps/FUDS-models/SMRFUDS-Test Two-19.svg}
            \caption{Testing against US06 profile}
            \label{subfig:US_diff_prof_compare}
        \end{subfigure}
        \caption{Different initial periods of model validation}
        \label{fig:diff_prof_compare}
    \end{figure}
    % \begin{itemize}
    %     \item To verify the efficiency, model also was compared against the two other profiles cycling profiles. \\
    %     \item \textbf{model has tendency to put a lot of weight into the Voltage. This way. weights will fall into SoC instead to preserve longer dependency.} \\
    %     \item \textit{Different comparison, plots. Relative timing and what mechanism in implementation with lists and tensors increased it. Accuracies.} \\
    %     \item \textbf{Comaprison of different Windows size. Time it takes to compute that is too long. Need to overcome it, somehow.} \\
    % \end{itemize}
    \begin{figure}
        \centering
        \includesvg[width=\linewidth]{III_Conclussion/Models/Sadykov2021-30steps/FUDS-models/SMRFUDS-performance.svg}
        \caption{Accuracy evolution over training process for Mean Average Error (MAE) and Root Mean Squared Error (RMSE)}
        \label{fig:res_performance}
    \end{figure}
\subsection{Optimal output step size}
    The optimal size of the output steps for better capturing the State of Charge was determined through experimental training. Table~\ref{tab:out_steps} outlines the training iterations of each driving profile until it reached accuracy below 2.5\%.
    \begin{table}[htbp]
        \centering
        \caption{Percentage accuracy evaluation with increasing output step size}
        \label{tab:out_steps}
        \begin{tabular}{ p{1.5cm} || p{1.5cm} p{1.5cm} p{1.5cm} || p{1.5cm}  }
            \hline
            Steps & DST & US06 & FUDS & Epochs \\
            \hline
            10 & 2.94 & $\geq\sim 15$ & $\geq\sim 15$ & 8 \\
            15 & 0.90 & 10.67  & 11.10  & 8 \\
            20 & |    & 0.68   &  4.25  & 10 \\
            25 & |    & |      &  3.48  & 25 \\
            30 & |    & |      &  0.59  & 19 \\
            \hline
        \end{tabular}
    \end{table}
    
    %
    %
    The Dynamic Stress Test (DST) has a stable trend of current consumption and regenerative process.
    Therefore it did not take long to capture the behaviour of the charge.
    A single cycle on subfigure~\ref{subfig:res_DST} provides an example of accurate feed-forward charge estimation with output steps between 10-15 samples.
    The US06 and FUDS have the more aggressive current usage, leading to trendier voltage readings.
    As per subfigures~\ref{subfig:res_US} and~\ref{subfig:res_FUDS}, it took twice more output samples to achieve similar accuracy to DST based model.
    Any further increase can capture even more complicated scenarios, leading to a longer training time.
    However, the epoch per training model to produce the best fit decreased with every added step.
    
    %
    %
    An optimal step is required for running the training process on the entire dataset with all three profiles as inputs to prepare the deployment model.
    Therefore, all other results demonstrations will use 30 steps as the most optimal and report any other results based on that parameter.
    \begin{figure}[htbp]
        \centering
        % DST based tests
        \begin{subfigure}[b]{0.325\textwidth}
            \centering
            \includesvg[width=\linewidth]{III_Conclussion/Models/Sadykov2021-15steps/DST-models/SMRDST-FF-8.svg}
            \caption{Best DST based model after 8 iterations with 15 steps}
            \label{subfig:res_DST}
        \end{subfigure}
        \hfill
        \begin{subfigure}[b]{0.325\textwidth}
            \centering
            \includesvg[width=\linewidth]{III_Conclussion/Models/Sadykov2021-20steps/US06-models/SMRUS06-FF-10.svg}
            \caption{Best US06 based model after 10 iterations with 20 steps}
            \label{subfig:res_US}
        \end{subfigure}
        \hfill
        \begin{subfigure}[b]{0.325\textwidth}
            \centering
            % \includesvg[width=\linewidth]{III_Conclussion/im_best/FUDS-19.svg}
            \includesvg[width=\linewidth]{III_Conclussion/Models/Sadykov2021-30steps/FUDS-models/SMRFUDS-FF-19.svg}
            \caption{Best FUDS based model after 19 iterations with 30 steps}
            \label{subfig:res_FUDS}
        \end{subfigure}
        \caption{Best training results over 3 different driving profiles}
        \label{fig:Models_res}
    \end{figure}
\section{Conclusion} \label{sec:conclussion}
What was the goal, what was achieved and what will be applied for On-board computer 
\section{Acknowledgements} \label{sec:acknowledgements}
The research was undertaken through funding from the Autonomous...
\textcolor{red}{Following commented loop needs to be replaced with single huge fugire demonstrating performance evolution relative to other profiles}
% \foreach \i in {1,2,...,20}{
%     \begin{figure}[htbp]
%         \centering
%         \includesvg[width=\linewidth]{III_Conclussion/images/\i.svg}
%         \caption{Epoch/Iteration \i}
%     \end{figure}
% }


\bibliography{bibs}

\end{document}