\documentclass[a4paper,fleqn]{cas-dc}
% *** GRAPHICS RELATED PACKAGES ***
%
% \usepackage[pdftex]{graphicx}
% Adds SVG Support out of cls file
\usepackage{svg}
% *** SPECIALIZED LIST PACKAGES ***
%
\usepackage{algorithm, algorithmic}
% *** SUBFIGURE PACKAGES ***
\usepackage[caption=false,font=normalsize,labelfont=sf,textfont=sf]{subfig}
% *** FLOAT PACKAGES ***
%
%\usepackage{fixltx2e}
% \usepackage{stfloats}
% \usepackage{dblfloatfix}

% *** PDF, URL AND HYPERLINK PACKAGES ***
%
\usepackage{url}
% Adding if execution to define between thesis and article
\usepackage{ifthen}
\newboolean{thesis}
\setboolean{thesis}{false}
% lscape.sty Produce landscape pages in a (mainly) portrait document.
\usepackage{pdflscape}
% Fancy table related trick
% \usepackage{multirow}
% correct bad hyphenation here
% \hyphenation{op-tical net-works semi-conduc-tor IEEE-Xplore}
% updated with editorial comments 8/9/2021
% \usepackage[hidelinks]{hyperref}
% \usepackage{orcidlink}

% \usepackage[authoryear,longnamesfirst]{natbib}
\usepackage[numbers,square]{natbib}
% \let\cite\citep

%%%Author macros
\def\tsc#1{\csdef{#1}{\textsc{\lowercase{#1}}\xspace}}
\tsc{WGM}
\tsc{QE}

\begin{document}
\let\WriteBookmarks\relax
\def\floatpagepagefraction{1}
\def\textpagefraction{.001}

% Short title
\shorttitle{Feed-Forward State of Charge Estimation Using Time-Series Machine Learning Prediction With Autoregressive Models}

% Short author
\shortauthors{Marat Sadykov, Sam Haines, Geoffrey Walker and David William Holmes}

% Main title of the paper
\title[mode=title]{Feed-Forward State of Charge Estimation Using Time-Series Machine Learning Prediction With Autoregressive Models}

% Title footnote mark
% eg: \tnotemark[1]
% \tnotemark[1,2] 

% Title footnote 1.
% eg: \tnotetext[1]{Title footnote text}
% \tnotetext[1]{<tnote text>} 

% \thanks{Manuscript received March 19, 2023; revised April ??, 2023.}

% The paper headers
% \markboth{IEEE Transactions on Intelligent Vehicles,~Vol.~8, No.~10, November~2023}%
% {Sadykov \MakeLowercase{\textit{et al.}}: Feed-Forward State of Charge Estimation Ising Time-Series Machine Learning Prediction With Autoregressive Models on Lithium-Ion Batteries}

% First author
%
% Options: Use if required
% \author[1,3]{Author Name}[type=editor,
%       style=chinese,
%       auid=000,
%       bioid=1,
%       prefix=Sir,
%       orcid=0000-0000-0000-0000,
%       facebook=<facebook id>,
%       twitter=<twitter id>,
%       linkedin=<linkedin id>,
%       gplus=<gplus id>]
%* Author 1
\author[1]{M. Sadykov}[type=editor,
      auid=000,
      bioid=1,
      prefix=Sir,
      role=Researcher,
      orcid=0000-0002-6436-7069]

% Corresponding author indication
\cormark[1]

% Footnote of the first author
\fnmark[1]

% Email id of the first author
% \ead{sadykovmarat@pm.me}

% URL of the first author
% \ead[url]{www.qut.edu.au/1}

% Credit authorship
% eg: \credit{Conceptualization of this study, Methodology, Software}
\credit{Conceptualisation of this study, Methodology, Software, Validation, Formal Analysis, Investigation, Resources, Data Curation, Original draft preparation, Editing and Data Visualisation}

% Address/affiliation
\affiliation[1]{
            organization={School of Mechanical, Medical and Process Engineering (MMPE)},
            addressline={Queensland University of Technology~(QUT)}, 
            city={Brisbane},
            postcode={4000}, 
            state={QLD},
            country={Australia}}

\author[1]{Sam Haines}[
      style=chinese]
\credit{Conceptualisation of this study and Resources acquisition}
\author[2]{Geoffrey Walker}[
      style=chinese,
      orcid=0000-0001-8137-9507]
\credit{Work validation, article review and editing, and supervision}
\author[1]{David William Holmes}[
      style=chinese,
      orcid=0000-0002-2970-9158]

% Corresponding author indication
\cormark[1]

% Footnote of the second author
\fnmark[1]

% Email id of the second author
\ead{d.holmes@qut.edu.au}

% URL of the second author
% \ead[url]{www.qut.edu.au/1}

% Credit authorship
\credit{Conceptualisation of this study, Methodology, Validation, Formal Analysis, article review and editing, Supervision, Project administration and Funding acquisition}

% Address/affiliation
\affiliation[2]{
            organization={School of Electrical Engineering \& Robotics},
            addressline={Queensland University of Technology~(QUT)}, 
            city={Brisbane},
            postcode={4000}, 
            state={QLD},
            country={Australia}}

% Corresponding author text
\cortext[1]{Corresponding author.}

% Footnote text
\fntext[1]{These authors contributed equally to this work.}

% For a title note without a number/mark
%\nonumnote{}

\begin{abstract}[S U M M A R Y]
  This paper presents a Recurrent Neural Network (RNN) model for the State of Charge (SoC) estimation for an Electric Vehicle's Lithium-Ion battery.
  A new Autoregressive modification is proposed to improve upon earlier published Machine Learning models to enhance the prediction accuracy and make results less sensitive to varying usage conditions.
  By building upon earlier published findings, assessing several models ability to estimate SoC using a history window of sensory data (Sadykov \textit{et al}~\cite{sadykov_practical_2022}), this work proposes, implements, and reports its methods of improving accuracy by introducing the State of Charge as part of sensory inputs in a Feed-Forward manner.
  The method is based on the Autoregressive implementation of RNN model training, where successively calculated SoC values are used as input in the subsequent predictions, but the potential inaccuracies with those predictions that might otherwise cause significant and rapid divergence are accounted for within a modified recursive training procedure.
  As a result, with a slight increase in training time, the accuracy of the output prediction for three different realistic charge-discharge driving profiles doubled compared to the two best previously published RNN models.
  Overall, the test results have demonstrated the usability of the current model in actual driving scenarios, making such models a viable replacement to estimation approaches used in electric vehicle battery management systems.
\end{abstract}

% Research highlights
% \begin{highlights}
%   \item 1
%   \item 2
%   \item 3
% \end{highlights}

% Note that keywords are not normally used for peerreview papers.
\begin{keywords}
  Autoregressive models \sep
  % Lithium iron phosphate battery ($LiFePO_{4}$),
  Lithium-Ion Battery (Li-Ion) \sep
  Long Short-Term Memory Recurrent Neural Networks (LSTM-RNNs) \sep
  State of Charge (SOC) estimation \sep
  TensorFlow (TF)
\end{keywords}

% make the title area
\maketitle

\section{Introduction}\label{sec:introduction}
%
%
The past decade has seen a rapid growth of the market for Electric Vehicles (EVs), with all major auto makers currently investing heavily in EV platform development~\cite{iea_global_2023}, and the technology broadly expected to overtake and become equally or even more profitable than the internal combustion engine by around 2025~\cite{baik_making_2019}.
The promise of a clean, environmentally friendly transport future is an attractive one, but there remains significant work to be undertaken to increase battery range, battery lifespan, decrease cost, decrease weight, improve end of life management, and other key issues, to fully realise the sustainable potential of EV systems.
One area that remains a barrier to more sustainable battery utilisation is the development of a method to accurately estimate how much remaining charge is in an EV's battery in real-time.
In contrast to fuel-based vehicles where measuring volume of fuel remaining is a trivial task, the effective \textcolor{blue}{charge percentage} of a battery depends on multiple factors and can be different at different temperatures and conditions~\cite{xing_state_2014}, and change significantly over time due to ageing, damage, and other influences~\cite{johansson_neural_2018}.

%
%
While remaining an open research question, accurate battery SoC estimation is critical in assessing a batteries' performance, facilitating accurate estimates of the remaining range, ensuring the protection of battery health, and enabling the highest level of overall battery utilization~\cite{yamin_embedded_2014}.
The determination of SoC is typically carried out in a vehicle's Battery Management System (BMS), and the logic used in current systems employ techniques like Coulomb Counting (CC) that integrates measured current over time to estimate charge usage~\cite{robust_SoC}, or battery modelling based on equivalent circuits and using voltage, current, and other sensor data to approximate the battery behaviour~\cite{6953745,ng_enhanced_2009,robust_SoC}.
These approaches are simple to implement but are susceptible to inaccuracies due to coulombic inefficiencies~\cite{Smith_2010}, temperature variability~\cite{xing_state_2014}, non-linear multi-variable dependence of response ~\cite{hansen_support_2005,anton_battery_2013,he_state_2014}, and other issues.
More recently, computer intelligence or machine learning-based models have been proposed that use statistical methods and multi-dimensional data fitting based on battery cycle training data (usually voltage, current, and temperature measurements), as a way to improve the accuracy of SoC estimation and account for the many non-linear behaviours of a typical battery in a phenomenological way~\cite{hansen_support_2005,anton_battery_2013,he_state_2014}.
Methods employed include Fuzzy logic~\cite{malkhandi_fuzzy_2006}, Support Vector Machines~\cite{hansen_support_2005, anton_battery_2013}, and \textcolor{blue}{RNN}~\cite{song_lithium-ion_2018,Chemali2017,mamo_long_2020,jiao_gru-rnn_2020,xiao_accurate_2019,javid_adaptive_2020,zhang_deep_2020}.
While these methods have shown great promise in laboratory conditions (see for example~\cite{jiao_gru-rnn_2020}), there remains significant work to be done before they represent a viable option for the next generation of on-car BMS circuitry.

%
%“what are all the models applied to SoC, and why have you chosen LSTM”
% A wide variety of machine learning approaches have been used for SoC prediction in the literature.
% [Methods A, B, C] [refs] have been applied to [something] with advantages of […], and observed challenges […].
% Alternatively [methods D, and E] [refs] have addressed many of these issues, but also been found to […good or bad…].
% From all of these methods, LSTM […insert advantages…], while avoiding some of the drawbacks of other methods like […things it doesn’t suck at…].
% Also, LSTM is one of the simpler implementations that enables scaffolding of more advanced learning sequencing.
% The process implemented in this work could be applied to other more sophisticated ML algorithms also, but this is left outside the scope of this paper.”
%
Of the computer intelligence approaches employed in SoC estimation, RNNs were the most commonly applied.
Multiple examples employing time-series based models have been published, specifically: approaches based on the Long Short-Term Memory (LSTM) method~\cite{Chemali2017,mamo_long_2020,zhang_deep_2020} and the Gated Recurrent Unit (GRU) method~\cite{song_lithium-ion_2018,jiao_gru-rnn_2020,xiao_accurate_2019,javid_adaptive_2020}.
\textcolor{blue}{
Models used can be either categorised as stateless~\cite{Chemali2017,javid_adaptive_2020,zhang_deep_2020}, relying on a short dependency of history data of around a few minutes, applied at any point of cycle time but can suffer from drift if previous estimates are inaccurate, or stateful~\cite{song_lithium-ion_2018,jiao_gru-rnn_2020,xiao_accurate_2019} that preserve charge estimation from the beginning of the discharge cycle, being less sensitive to drift, while more susceptible to changes in condition like temperature.
% Models used can be either categorised as stateless ~\cite{Chemali2017,javid_adaptive_2020,zhang_deep_2020} relying on a short dependency of history data of around a few minutes, or stateful~\cite{song_lithium-ion_2018,jiao_gru-rnn_2020,xiao_accurate_2019} that preserve charge estimation from the beginning of the discharge cycle. 
% Stateless models can be applied at any point of cycle time but can suffer from drift if previous estimates are inaccurate, whereas stateful are less sensitive to drift, but tend to be more susceptible to changes in condition like temperature.
% Most of the work focussed in stateless~\cite{Chemali2017,javid_adaptive_2020,zhang_deep_2020} and statefull~\cite{song_lithium-ion_2018,jiao_gru-rnn_2020,xiao_accurate_2019} models to predict SoC on battery data attempting to either rely on a short dependency of history data within few minutes, making the models applicable at any point of cycle time, or preserve from the beginning of the discharge cycle only, resulting in more susceptible results with condition change, mainly temperature.
Improvements to conventional RNN models that have been applied to SoC estimation include modifying the models with additional logic (like an Attention layer~\cite{mamo_long_2020}) to improve accuracy, or implementing modified optimisation procedures like RoAdam~\cite{javid_adaptive_2020},to improve the overall training process through smoother fitting lines.
% Others attempted to modify the models either with an additional logic (an Attention layer~\cite{mamo_long_2020}), landing better end accuracy results or modified optimisation procedures, like RoAdam~\cite{javid_adaptive_2020}, and improve the overall training process with smoother fitting lines.
In an earlier work~\cite{sadykov_practical_2022}, we evaluated the ability of multiple implementations of RNN models to estimate the SoC of a lithium iron phosphate battery across multiple different simulated drive cycles.}
% In an earlier work~\cite{sadykov_practical_2022}, we evaluated the ability of multiple implementations of RNN models to estimate the SoC of a lithium iron phosphate battery across multiple different simulated drive cycles.}
The research found that a model trained on one driving scenario was effective in accurately reproducing the full battery utilisation of that specific driving behaviour but was poor at extrapolating to different drive cycle behaviours that were outside the training set (e.g., training on DST, and predicting based on US06 and FUDS\footnote{DST, US06, and FUDS are examples of dynamic drive cycles that are used to evaluate internal combustion engine (ICE) vehicle fuel consumption, emissions and now the discharge and SoC characteristics of EV batteries~\cite{castillo_18_2015}. They involve dynamic charge-discharge histories that are applied to a battery that are meant to simulate a car driving around an urban environment (i.e. accelerating and decelerating with regenerative breaking, plus full charge cycles).}), with inaccuracy increasing by at least double.
The best model identified was the LSTM with Attention Layer~\cite{mamo_long_2020}, which had an SoC estimation accuracy of 1.82\% on the same FUDS drive cycle as trained on, and an accuracy of 3.73\% and 5.27\% for the DST and US06 drive cycles that were not in the training set.
% The advantage of LSTM over GRU has not been discovered, and the simplicity of gates or their numbers did not yield any noticable results.
In a practical implementation of SoC estimation (i.e. on a real vehicle), the ability to accurately predict SoC outside a given training set is critical, as all people will use and drive their vehicles in a slightly different way.
It is this feature that was the primary evaluation in~\cite{sadykov_practical_2022}, and the main motivation for the new model proposed in this work.

%
%
A key difference between the application of machine learning to many other non-linear systems (e.g., financial market predictions or analysis of the weather) and its application to SoC estimation is that the primary focus of the prediction in batteries, i.e. SoC, cannot be directly measured in real-time\footnote{SoC can be measured for a battery directly, but this requires a long time settling tests that may take minutes or hours at a steady battery state for it to reach equilibrium and measurement to be taken~\cite{ali_towards_2019}. This is infeasible for a real-time SoC measurement approach for EV applications.}.
As such, SoC is typically not used as an input feature to training or prediction (whereas the current market or weather state are used to improve the next prediction in the other examples).
The absence of a real-time “ground truth” of SoC is one of the features that makes SoC prediction so challenging.
The risk of using predicted SoC as an input parameter to any computer intelligence model is evident because minor initial errors may rapidly cause a divergent solution.
% However, if this risk could be managed, there is potential to improve significantly the accuracy of such methods for SoC prediction.
% Several options exist.
\textcolor{blue}{However, the risk could be reduced by using initial SoC from other means, like 3-feature based models discussed in previous work~\cite{sadykov_practical_2022}, or by employing the charge feature to those models utilising an autoregressive technique~\cite{time_2020} to avoid the propagation of prediction error.}

%
%
% The first is to use an initial SoC estimation from other means (e.g., from coulomb counting).
% This may, however, be susceptible to the same limitations experienced by the chosen method.
% The second is to employ an RNN model to propagate the output value of the charge as an input to the next prediction but utilise an autoregressive technique~\cite{time_2020} to avoid the propagation of prediction error.
% This approach should be more reasonable and practical than attempting to manually distribute weights between features or manipulate with an introduction of noise to input data to make output less sensitive to mis-accuracies.
%The example of the technique from documentation~\cite{time_2020} provided a reasonable way to approach.
% Such a technique has been used in~\cite{time_2020} and provides a good potential approach.
% However, instead of looking towards the future outputs with initial inputs only, the set of predetermined sensory data will be provided to the model in our case.
\textcolor{blue}{This paper will carry on the previous findings to overcome identified issues in the estimation of full-cycle battery utilisation and propose a new method, which has not been researched deeply enough.
A novel method for implementing a SoC estimation training loop will be presented based on the autoregressive technique, which uses a 4-feature input (current, voltage, temperature, and SoC), with a history of 8 minutes and 20 seconds usage (500 samples at 1Hz), and increases SoC prediction accuracy, while avoiding the accumulation of prediction error that might otherwise make a feed-forward based model infeasible.
That way, the model will be forced to consider the accuracy of its charge prediction before loss calculation.}

%
%
The rest of this paper is organised as follows: a methodology for an RNN model is discussed in Section~\ref{sec:layer}.
The details of how auto-regression has been utilised are in Section~\ref{sec:feed}.
% Subsections 4.1 and 4.2 separate model validation points and parameter estimation processes.
Section~\ref{sec:results} summarises the investigation results following the evaluation approach developed in~\cite{sadykov_practical_2022} and compares the degree of improvement over traditional RNN methods.
Finally, Section~\ref{sec:conclussion} concludes the research by outlining several observations, which may require separate consideration.
% Most were isolated to closed scenarios with provided data or from battery cycling machines.
% The most promising approach to improve a model and make it more universal is to increase complexity. While some introduced deeper layer network, others added additional mechanisms to those already used.
% \hfill mds 
% \hfill August 26, 2015

\section{Method Development} \label{sec:layer}
Before the development of the new approach is detailed, we need to outline how the methods and the data will be assessed against the testing methodology.
% \pdfcomment{
%   ALTERNATIVELY:
%   The following subsection/part will describe battery data and performance metrics used in similar methods, structure and modifications of the researched Time-series model, and summaries a newly proposed method to improve the overall accuracy.
% }
Subsection~\ref{subsec:testing} outlines what and how battery data is used and reported for performance metrics against other methods.
Subsection~\ref{subsec:lstm} presents the type, structure and method of the time-series machine learning models which will be utilised throughout the research, along with input and output data management and optimisation strategy.
Subsection~\ref{sec:feed} provides the full summary of how the new method is implemented from the procedural perspective and why it is considered viable to improve the overall accuracy.
\subsection{Testing methodology}~\label{subsec:testing}
  The battery cycling data were obtained from the Battery Research Group of the
Center for Advanced Life Cycle Engineering (CALCE) Group
at the University of Maryland~\cite{noauthor_calce_2017}.
The temperatures of 20,25,30 degrees over three driving profiles were selected as training and testing groups.
Each temperature was taken from three driving profiles: Dynamic Stress Test (DST), Highway driving US06 and Federal-Urban driving schedule (FUDS)
The FUDS driving scheduler acts as a training set, then the other two DST and US06 for testing.
The choice of validation sets was due to differences in the current consumption.
If DST has a constant variation of the current amps over time, the US06 uses an aggressive mechanism, similar to the FUDS.
The ability to accurately capture both driving mechanisms with minor error acts as a validation mechanism and justifies the value of the technique.

%
%
Report resutls based on error degradation pver Mean Average Error and Root MEan Squared Errors.
Choosing the most optimal will be the test subject.
\begin{table}[htbp]
    \renewcommand{\arraystretch}{1.3}
    \caption{Accuracy result summary for entire training datasets.}
    \centering
    \label{tab:acc-results2}
\resizebox{\linewidth}{!}{
\begin{tabular}{ c| l| c c c| c c c |c c c}
    \hline\hline \\[-4mm]
% Columns setup
    \multirow{3}{1em}{\#} &
    \multirow{3}{3em}{Trained} &
    \multicolumn{9}{c}{Tested} \\
    \cline{3-11}
    & & 
    \multicolumn{3}{c|}{DST} &
    \multicolumn{3}{c|}{US06} &
    \multicolumn{3}{c}{FUDS} \\
    \cline{3-11}
     & & MSE(\%) & RMSE(\%) & $R^{2}$(\%) & MSE(\%) & RMSE(\%) & $R^{2}$(\%) & MSE(\%) & RMSE(\%) & $R^{2}$(\%) \\
    \hline
    % Content
    %Chemali2017
      & DST & 2.77 & 3.52 & 98.71 & 2.86 & 3.93 & 98.34 & 3.28 & 4.62 & 97.66 \\ 
    1 & US06 & 5.97 & 7.97 & 93.39 & 3.37 & 4.14 & 98.15 & 5.38 & 6.93 & 94.73 \\ 
      & FUDS & 5.03 & 7.26 & 94.51 & 4.02 & 6.07 & 96.04 & 1.95 & 2.85 & 99.11 \\ 
    \hline
      & DST & 2.86 & 3.60 & 98.65 & 2.91 & 3.79 & 98.46 & 3.73 & 5.18 & 97.06 \\ 
    3 & US06 & 5.98 & 8.26 & 92.90 & 3.35 & 4.11 & 98.19 & 5.27 & 6.84 & 94.87 \\ 
      & FUDS & 5.33 & 7.25 & 94.53 & 3.61 & 5.53 & 96.71 & 1.82 & 2.51 & 99.31 \\ 
    % GelarehJavid2020
    % \hline
    %   & DST & 2.89 & 3.61 & 98.65 & 3.82 & 5.38 & 96.88 & 4.11 & 5.51 & 96.67 \\ 
    % 4 & US06 & 6.19 & 8.57 & 92.35 & 3.30 & 4.12 & 98.17 & 5.42 & 6.82 & 94.91 \\ 
    %   & FUDS & 5.74 & 7.49 & 94.16 & 4.03 & 5.75 & 96.44 & 1.58 & 2.28 & 99.43 \\ 
    \hline
      & DST & 0.81 & 1.10 & 99.87 & 1.92 & 2.71 & 99.21 & 1.77 & 2.35 & 99.40  \\
    6  & US06 & 1.49 & 2.02 & 99.57 & 0.89 & 1.06 & 99.88 & 1.28 & 1.94 & 99.59  \\
      & FUDS & 2.10 & 2.94 & 99.10 & 1.28 & 1.87 & 99.62 & 0.50 & 0.68 & 99.95  \\
    \hline\hline
\end{tabular}
}
\end{table}
%
%
BEst one per each iteration will be summarised in a table and then compared against two acccuracy plots as per following example.
\begin{figure*}[htbp]
    \centering
    %%%%%%%%%%%%%%%%% DST based tests %%%%%%%%%%%%%%%%%
    \begin{subfigure}[b]{0.325\textwidth}
        \centering
        \includesvg[width=\linewidth]{II_Body/images/M6-history-DST-mae.svg}
        \caption{Average training and testing MAE history average of 10 attempts}
    \end{subfigure}
    \hfill
    \begin{subfigure}[b]{0.325\textwidth}
        \centering
        \includesvg[width=\linewidth]{II_Body/images/DST-6-train.svg}
        \caption{Validation on a single cycle of SoC estimation average of 10 attempts at 25\textdegree{}C}
    \end{subfigure}
    \hfill
    \begin{subfigure}[b]{0.325\textwidth}
        \centering
        \includesvg[width=\linewidth]{II_Body/images/DST-6-test.svg}
        \caption{Testing on two cycles of US06 and FUDS profiles average of 10 attempts}
        \label{subfig:Model-1res-DSTvsFUDS}
    \end{subfigure}
    %%%%%%%%%%%%%%%%% US06 based tests %%%%%%%%%%%%%%%%%
    \begin{subfigure}[b]{0.325\textwidth}
        \centering
        \includesvg[width=\linewidth]{II_Body/images/M6-history-US06-mae.svg}
        \caption{Average training and testing MAE history average of 10 attempts}
    \end{subfigure}
    \hfill
    \begin{subfigure}[b]{0.325\textwidth}
        \centering
        \includesvg[width=\linewidth]{II_Body/images/US06-6-train.svg}
        \caption{Validation on a single cycle of SoC estimation average of 10 attempts at 25\textdegree{}C}
    \end{subfigure}
    \hfill
    \begin{subfigure}[b]{0.325\textwidth}
        \centering
        \includesvg[width=\linewidth]{II_Body/images/US06-6-test.svg}
        \caption{Testing on two cycles of US06 and FUDS profiles average of 10 attempts}
    \end{subfigure}
    % %%%%%%%%%%%%%%%%% FUDS based tests %%%%%%%%%%%%%%%%%
    \begin{subfigure}[b]{0.325\textwidth}
        \centering
        \includesvg[width=\linewidth]{II_Body/images/M6-history-FUDS-mae.svg}
        \caption{Average training and testing MAE history average of 10 attempts}
    \end{subfigure}
    \hfill
    \begin{subfigure}[b]{0.325\textwidth}
        \centering
        \includesvg[width=\linewidth]{II_Body/images/FUDS-6-train.svg}
        \caption{Validation on a single cycle of SoC estimation average of 10 attempts at 25\textdegree{}C}
    \end{subfigure}
    \hfill
    \begin{subfigure}[b]{0.325\textwidth}
        \centering
        \includesvg[width=\linewidth]{II_Body/images/FUDS-6-test.svg}
        \caption{Testing on two cycles of DST and US06 profiles average of 10 attempts}
    \end{subfigure}
    \caption{Model 6: Stateless LSTM with 30 steps of Autoregression.}
    \label{fig:Model-6res}
\end{figure*}
\clearpage

%
%
Figure~\ref{fig:plot_demo} shows an example of accuracy evaluation, where actual State of Charge compared with prediction.
The filled area below the plot captures the error Absolute Error Difference between two lines, as per the Equation~\ref{eq:abs-error}.
For the purpose of comparioson, all Figures with two subplots were distiguished from each other by prediction line color, as per early Figure~\ref{fig:regular_tr}.
The left subfigures~\textit{a} with red output line referes to the input taken directrly from the table, including the perfect SoC history.
The right subfigures~\textit{b} with yellow prediction refers to Feed-Forward method, where only Voltage, Current and Temperature were taken from the table through the testing.
Only the first windows of samples were taken from other sources and considered ideal, either due to be the initial fully charge or discharge state, or estimated with other methods.
Everything following prediction is based on the early produced output of the same model.
No $R^2$ has been produced on the Feed-forward prediction, since it is mathematically not feasible at current state due to the mix of multiple outputs.
\begin{figure}[ht]
    % RMSE equation: RMS = (tf.keras.backend.sqrt(tf.keras.backend.square(y_test_one[::,]-PRED)))
    \centering
    \includesvg[width=\columnwidth]{II_Body/images/plot-example.svg}
    \caption{Accuracy plot demonstration.}
    \label{fig:plot_demo}
\end{figure}
\begin{equation}
    \textbf{ABS error}  = \sqrt{(Actual-Prediction)^2}
    \label{eq:abs-error}
\end{equation}
\subsection{LSTM estimation}~\label{subsec:lstm}
  Time-series prediction of a neural model relies on the input history of equally distributed samples.
\ifthenelse {\boolean{thesis}}{As discussed in earlier chapters, the LSTM model is a recurrent neural network designed to solve the vanishing gradient problem by remembering (preserving) the long dependencies \cite{rasifaghihi_predictive_2020}.}
{The LSTM model is a recurrent neural network designed to solve the vanishing gradient problem by remembering (preserving) the long dependencies~\cite{rasifaghihi_predictive_2020}.}
The cells inside the model act as memory units to preserve the dependence.
Therefore, the output is closely dependent on the previous input samples.
Unlike the normal RNN, and the more modern version GRU, LSTM has a more complicated structure constructed from several logical gates~\cite{LSTM_Hochreiter1997}.
It is the most widely used type of model.
\ifthenelse {\boolean{thesis}}{Chapter~\ref{cha:Analysis} provides a summary of the LSTM cell logic, with corresponding equations explaining the gates logic in detail.} 
{\mbox{Figure~\ref{fig:LSTM-cell2}} provides a summary of the cell logic.
It utilises three gates: forget $f_t$, input $i_t$ and output $o_t$, \mbox{Equation~\ref{eq:LSTM-gates2}}.
The decisions are based around sigmoid $\sigma$ function~\ref{eq:sigmoid2}.
With default $tanh$ as activation function, \mbox{Equation~\ref{eq:LSTM-output2}} describes the procedure for cell state update and further propagation.
Output variables $h_t$ and $c_t$ represent memory cell output and the cell state at timestamp $t$.
\begin{equation}
    \sigma(x) = \frac{1}{1+e^{-x}}
    \label{eq:sigmoid2}
\end{equation}
\begin{figure}[htbp]
    \centering
    \includegraphics[width=\linewidth]{II_Body/LSTM/images/LSTM.jpg}
    \caption{Long Short-Term Memory Cell}
    \label{fig:LSTM-cell2}
\end{figure}
\begin{equation}
    \begin{split}
        f_t &= \sigma \left(W_f \left[h_{t-1}, x_t \right] + b_f \right) \\
        i_t &= \sigma \left(W_i \left[h_{t-1}, x_t \right] + b_i \right) \\
        o_t &= \sigma \left(W_o \left[h_{t-1}, x_t \right] + b_o \right) \\    
    \end{split}
    \label{eq:LSTM-gates2}
\end{equation}
\begin{equation}
    \begin{split}
        c_t &= f_t c_{t-1}+i_t \times tanh \left(W_c \left[h_{t-1}, x_t \right] + b_c \right) \\
        h_t &= o_t*tanh \left(c_t \right)
    \end{split}
    \label{eq:LSTM-output2}
\end{equation}
}

%
%
The LSTM model has been used widely in stock-price prediction or weather forecasting. 
%? add references if I feel to it
However, unlike State of Charge estimation, which commonly uses $V$, $I$, and $T$ as inputs, those methods utilise the output feature as an input to the subsequent prediction to propagate results further and calculate the time before a critical event occurrence.
Besides, methods like weather forecasting for a week are not limited by lacking output data, since the searched criteria are always known or will be known once they happen, allowing updates and improvements of follow-up predictions.
On the contrast, a battery's actual State of Charge cannot be directly determined or measured making verification against previously-made predictions difficult without additional battery modelling techniques or laboratory equipment.
%, not to mention having it integrated into an Electric Vehicle's accumulator.
It can be determined with a proper battery cycler, performing a set of pulse tests, but this is infeasible for practical applications.
%utilising a battery and affecting its charge and remaining life.
% (As opposed to the SoC estimate, where getting actual values to require a battery cycler capable ... ) 
%%%%%%%%%%%
%Unlike the charge estimation, which can only output a single value based on a history of samples, they are not limited to ... . Therefore, it does not require the output as input since the truth will become known in due time.
As such, to include the charge in the process, SoC as a learning input is used, later making a predicted array of values used on testing.
This introduces potential issues for error accumulation with every evaluation that will be addressed.

%
%
The best way to use the performance of the stateless LSTM model is through training with a data windowing technique.
The NN model will receive a fixed set of equally-distributed time samples at each time prediction.
Every next forecast will shift the time window by a constant step $s$, until all possible combinations of time slices go through the model.
That approach is referred to as a stateless model, which only sees dependencies over input samples rather than preserving every received input, like in stateful implementations.
It also allows the order of the windows to be shuffled to avoid overfitting.
Since no dropout was applied before the model's output, a set of strategies has been applied to update the learning rate and rollback before early stopping to assist the fitting process.
\ifthenelse {\boolean{thesis}}{Chapter~\ref{cha:Analysis} at subsections~\ref{subsec:l-rate} and~\ref{subsec:t_model} explain those two methods, which have already proven to be effective at training SoC models.} 
{Early research on published methods has already utilised those two methods to assist in the models' training process [Sadykov, 1].
It involved a scheduled learning rate value update with every passing epoch for as long as the accuracy kept improving with every passing iteration, as well as assisting the models' recovery in the event of overfitting by double reduction of the value either until the models' return to the same minimal optimisation or finalising the optimum result reach.}
As a result, a NN model will learn dependency between a fixed amount of equally distributed time samples $n$ and yet be independent from the order of the inputs.
\ifthenelse {\boolean{thesis}}{\mbox{Figure~\ref{fig:Windowing}} adapts the earlier Figure~\ref{fig:Windowing3f}, demonstrating how the input dataset is constructed and ordered into a 3-dimensional dataset, with four features (Voltage, Current, Temperature and added initial SoC), 500 timestamps and around a hundred thousand samples to fit on.}
{\mbox{Figure~\ref{fig:Windowing}} demonstrates how the input dataset is constructed and ordered into a 3-dimensional dataset, with four features, 500 timestamps and around a hundred thousand samples to fit on.}
Due to the size of the windows, equivalent to 8 and a quarter minutes of a discharge process, no batching mechanism has been used to reduce computational load and avoid 4-dimensional matrix management.

%
%
\ifthenelse {\boolean{thesis}}{Similar to Chapter~\ref{cha:Analysis}, the mean and standard deviation has been used to normalise all data to speed up the training process.}
{The mean and standard deviation has been used to normalise all data to speed up the training process.}
The normalisation constant from training input samples has been used for all further validation and testing sets to ensure the right trends.
The state of charge narrowed between 0 and 1 to represent the percentage charge to two decimals.
\mbox{Table~\ref{tab:params}} highlights the parameters required to define the initial model, where $s$ defines output step size, which will be justified later in Section~\ref{sec:feed}.
A $\sigma$ function as an output justifies the charge normalisation between 0 and 1.
\ifthenelse {\boolean{thesis}}{The number of neurons has been selected based on the results made in Chapter~\ref{cha:Analysis}, Section~\ref{sec:AN:Results}.
Even though the number of neurons was kept as per the performance result table, only one layer has been utilised due to manual implementation of the model and the inability to validate the multilayer implementation correctness against some published or already-used models.
It was decided to stick to known approaches to validate the efficiency of the newly-proposed technique.
\begin{table}[ht]
    \renewcommand{\arraystretch}{1.3}
    \caption{Model structure and parameters}
    \centering
    \label{tab:params}
    \begin{tabular}{ l l l }
        \hline\hline \\[-4mm]
        Input     & $shape= \left( 1,500,4 \right)$ & $batch=1 $  \\
        \hline
        LSTM      & $activation= 'tanh'$ & $units=131$  \\
        \hline
        Dropout   & $0.0$ &   \\
        \hline
        Output    & $activation= \sigma\left(s, 1 \right)$ &   \\
        \hline\hline
    \end{tabular}
\end{table}
}
{The number of neurons has been selected based on the results made based on earlier discoveries of the most optimal hyperparameters set [Sadykov, 1].
Results were adapted to the new custom-implemented model to validate the technique's efficiency in a comparable way with already-tested approaches.
\begin{table}[ht]
    \renewcommand{\arraystretch}{1.3}
    \caption{Model structure and parameters}
    \centering
    \label{tab:params}
    \resizebox{\columnwidth}{!}{
    \begin{tabular}{ l l l }
        \hline\hline \\[-4mm]
        Input     & $shape= \left( 1,500,4 \right)$ & $batch=1 $  \\
        \hline
        LSTM      & $activation= 'tanh'$ & $units=131$  \\
        \hline
        Dropout   & $0.0$ &   \\
        \hline
        Output    & $activation= \sigma\left(s, 1 \right)$ &   \\
        \hline\hline
    \end{tabular}
    }
\end{table}
}

%
%
\ifthenelse {\boolean{thesis}}
{
The optimisation algorithm for the fitting process has been selected as the regular Adam method, which was highlighted in Chapter~\ref{cha:Analysis}, \mbox{Algorithm~\ref{alg:Adam}}, with the corresponding hyperparameters, \mbox{Table~\ref{tab:uni-hyperparams}}.
The selection of this optimiser is made to allow comparison to two early-created LSTM-based models which use the same optimiser in Chapter~\ref{cha:Analysis}.
}
{
The optimisation algorithm for the fitting process has been defined by Adam, \mbox{Algorithm~\ref{alg:copyAdam}}, with the corresponding hyperparameters on \mbox{Table~\ref{tab:newM-params}}.
%\textcolor{red}{Try to use Robust Adam instead, because why the hell not since I lost one month of my life to implement that cursed algorithm from Javids miss-typed notes? Complete this section with details as per Gareth Javid's implementation if RoAdam will be able to produce a faster fitting.}
\begin{algorithm}\captionsetup{labelfont={sc,bf}, labelsep=newline}
    \caption{Adaptive Moment Estimation (Adam) optimisation}
    \begin{algorithmic}[1]
        \STATE \textbf{Number of input samples} \\ $N\gets length(\textit{input data})$\\
        \STATE \textbf{Size of windows} \\ $S\gets length(V_{i..n})$\\
        \STATE \textbf{Output steps} \\ $O\gets length(V_{i..n})$\\
        \STATE Input: $x_n = [V_{i..n}, I_{i..n}, T_{i..n}, SoC_{(i-1)..(n-1))}]$ \\
        - Shape: $X = (N, S, 4)$
        \STATE Output:$y_n = [SoC_{(n-o)..n}] - $Shape:$Y = (N, O, 1)$
        \STATE Define Loss function: $L$ \\
                Get hyperparameters: $\alpha, \beta_1, \beta_2, \epsilon$
        \WHILE{$W_t \text{ does not converge}$}
        \STATE $t \gets t+1$
        \STATE $g_t \gets \nabla_W L_t (W_{t-1})$ \COMMENT{Obtain gradient}
        \STATE $m_t \gets \beta_1 m_{t-1}+(1-\beta_1) g_t $ \COMMENT{$1_{st}$ moment calculation}
        \STATE $\upsilon_t \gets \beta_2 \upsilon_{t-1}+ \left(1-\beta_2 \right)g^2_t $ \COMMENT{$2_{nd}$ moment calculation \label{alg:Adam-Line-2Moment}}
        \STATE $\hat{m_t} \gets \frac{m_t}{1-\beta^t_1}$ \COMMENT{Corrected $\hat{m_t}$}
        \STATE $\hat{\upsilon_t} \gets \frac{\upsilon_t}{1-\beta^t_2} $ \COMMENT{Corrected $\hat{\upsilon_t}$}
        \STATE $W_t \gets W_{t-1}- \alpha \frac{\hat{m_t}}{\sqrt{\hat{\upsilon_t}}+\epsilon} $ \COMMENT{Update parameters}
        \ENDWHILE
    \end{algorithmic}
    \label{alg:copyAdam}
\end{algorithm}
\begin{table}[htbp]
    \renewcommand{\arraystretch}{1.3}
    \caption{Optimiser Hyper-Parameters}
    \centering
    \label{tab:newM-params}
    \resizebox{\columnwidth}{!}{
    \begin{tabular}{ l l l l l l }
        \hline\hline \\[-3mm]
        $\alpha$ & $\beta_1 $ & $\beta_2$ & $\beta_3$ &  $\epsilon$ \\
        \hline
        Linear         &  &  &  & \\% 0.0000001
        Scheduler      & $0.9$ & $0.999$ & $0.999$ &$10^{-8}$ \\% 0.0000001
        (0.001-0.0001) &  &  &  & \\% 0.0000001
        \hline\hline
    \end{tabular}
    }
\end{table}
}  
\subsection{Development of Feed-Forward Autoregressive model} \label{sec:feed}
  The regular training procedure for a 4-feature model has a significant limitation. The model does not take into account the Feed-Forward application of the prediction output.
In the case of excellent input values, the output is expected to be within~1 \% miss accuracy, since from 500 of ideal SoC values estimation of 501 is a trivial task, not requiring a Neural Network to perform.
\begin{figure*}[htbp]
    \centering
    % DST based tests
    \begin{subfigure}[b]{\columnwidth}
        \centering
        \includesvg[width=\linewidth]{III_Conclussion/im_compare/FUDS-val-48.svg}
        % \includesvg[width=\linewidth]{III_Conclussion/Models/Sadykov2020/FUDS-models/SMFUDS-val-9.svg}
        \caption{Regular training process snapshot}
        \label{subfig:regular_tr}
    \end{subfigure}
    \hfill
    \begin{subfigure}[b]{\columnwidth}
        \centering
        % \includesvg[width=\linewidth]{III_Conclussion/Models/Sadykov2020/FUDS-models/SMFUDS-FF-9.svg}
        \caption{Feed-Forward validation process snapshot}
        \label{subfig:regular_ts}
    \end{subfigure}
    \caption{Comparison between training and testing accuracies of a 4-featured based model with a default training and testing loop}
    \label{fig:regular_tr}
\end{figure*}
% \begin{figure}[ht]%[htbp]
%     \centering
%     \includegraphics[width=0.7\linewidth]{II_Body/images/multistep_autoregressive.png}
%     \caption{documented way}
%     \label{fig:autoregressive}
% \end{figure}

%
%
Two subplots, \mbox{Figure~\ref{fig:regular_tr}}, demonstrate the prediction results of a four feature-based trained model against a single battery cycle of DST driving.
\mbox{Subfigure~\ref{subfig:regular_tr}} demonstrates the prediction based on the always known perfect State of Charge, opposite on \mbox{subfigure~\ref{subfig:regular_ts}} with only perfect initial, and every following sample gets fed-forward.
The loss axis has been dropped from plotting due to high inaccuracy.
\ifthenelse{\boolean{thesis}}{
    The implementation of this prediction method is presented in \mbox{Appendix~\ref{app:Feed-Forward}}.
}{}
It demonstrates how the appended charge output model accumulates the error with every dependant input in a single prediction.
If that output will be used for further prediction and the model keeps preserving the dependency, the miss accuracy value rises non-linearly.
The reason for that lies in the amount of weight the model places on the State of Charge input feature.
For a better weights balance, the training procedure must be modified to consider the possibility of an inaccuracy in the input charge data.
The diagram in \mbox{Figure~\ref{subfig:testing}} illustrates regular training and testing procedures for a model to produce output.
The implementation has been based on contributions from the original framework developers ~\cite{time_2020}.
Details have been attached to Appendix~\ref{app:AutoFeedback}.
\begin{figure*}[htbp]
    \centering
    % DST based tests
    \begin{subfigure}[b]{0.85\textwidth}
        \centering
        % \includegraphics[width=\linewidth]{II_Body/images/IMG_20210524_133103.jpg}
        \includegraphics[width=\linewidth]{II_Body/images/Autoregression-Training.png}
        \caption{Custom autoregressive training procedure}
        \label{subfig:testing}
    \end{subfigure}
    \hfill
    \begin{subfigure}[b]{0.85\textwidth}
        \centering
        % \includegraphics[width=\linewidth]{II_Body/images/IMG_20210524_133052.jpg}
        \includegraphics[width=\linewidth]{II_Body/images/Autoregression-Testing.png}
        \caption{Regular testing and validation procedure}
        \label{subfig:training}
    \end{subfigure}
    \caption{Comparison between training and testing accuracy of a 4-featured based model with a default training and testing loop.}
    \label{fig:training_testing}
\end{figure*}

%
%
The training procedure for the regular LSTM model must be modified to consider potential inaccuracy using the autoregression technique.
%\textcolor{red}{I do not feel comfortable referencing the documentation, but what choice do I have now.}
The diagram in \mbox{Subfigure~\ref{subfig:training}} demonstrates the procedure for the model call using autoregression.
Unlike regular LSTM, training and testing differentiate from each other.
If the testing procedure remained unchanged, the training performs multiple calls during a single-window sample processing.
Every new call outputs the results and feeds again into the same model, with one sample from each sensor.
Each output also contained a model state, containing the values stored in the cell, preserving dependency between model calls.
State output is used only for internal model processing.
Every output of every step has been stored as an array.
With a new approach, an optimiser will compare an array of predicted samples against the true values of the SoC.
This way meant to increase the model fit process.
The more output samples model returns during the training, the better the real-time prediction against aggressive driving profiles.
\mbox{Figure~\ref{fig:modefied_tr}} contains a similar test as without autoregression.
Even though the accuracy with tabled samples has decreased, its feed-forward prediction accuracy has significantly increased.
\begin{figure*}[htbp]
    \centering
    % \begin{subfigure}[b]{\columnwidth}
    %     \centering
    %     \includesvg[width=\linewidth]{III_Conclussion/Models/Sadykov2021-30steps/FUDS-models/SMRFUDSval-19.svg}
    %     \caption{Modified training process}
    % \end{subfigure}
    % \begin{subfigure}[b]{\columnwidth}
    %     \centering
    %     \includesvg[width=\linewidth]{III_Conclussion/Models/Sadykov2021-30steps/FUDS-models/SMRFUDS-FF-19.svg}
    %     \caption{Feed-Forward validation process}
    % \end{subfigure}
    \caption{Comparison between training and testing accuracies of a 4-featured based model with a modified training and default testing loop}
    \label{fig:modefied_tr}
\end{figure*}
\newpage

\section{Results} \label{sec:results}
\subsection{Accuracy comparison}
    The newly proposed training technique was compared against a similar implementation without a modified training loop.
    Both models were provided with ideal initial results, including the State of Charge.
    Every further prediction replaced the actual charge value and was used as an input in the following input set, along with actual Voltage, Current and Temperature.
    
    %
    %
    Several plots in Figure~\ref{fig:diff_prof_compare} outline demonstrates the process of model validation and testing.
    The model has been trained and validated against FUDS-profile, subfigure~\ref{subfig:FUDS_diff_prof_compare}, and tested again DST on subfigure~\ref{subfig:DST_diff_prof_compare} and US06 on subfigure~\ref{subfig:US_diff_prof_compare}.
    The plots are based on the latest output from the training loop.
    The entire training process has been logged, and every model is saved as a separate checkpoint to determine the most efficient model in both validation and testing.
    Figure~\ref{fig:res_performance} demonstrates selecting the best model, marking the iteration which produced the lowest error for all three profiles.
    This way, the selection process can determine the iteration at which the model reached the lowest error across all three profiles. 

    %
    %
    The offset in the validation can be explained by the amount of weight placed into the known SoC, unlike with 3-feature models, where voltages act as the primary characteristic.
    In the State of Charge estimation, the significant impact is affected by current, and since the State of charge is the function of current and time, the weight the properly applied to the correct feature over the training process.
    Any other training process appeared to be unnecessary and may lead to overfitting.
    Figure~\ref{fig:Models_res} demonstrates the best feed-forward prediction for all three profiles.
    % \begin{figure}[htbp]
    %     \centering
    %     % DST based tests
    %     \begin{subfigure}[b]{0.475\textwidth}
    %         \centering
    %         \includesvg[width=\linewidth]{III_Conclussion/im_time/train-iCharging.svg}
    %         \caption{Charging process}
    %     \end{subfigure}
    %     \hfill
    %     \begin{subfigure}[b]{0.475\textwidth}
    %         \centering
    %         \includesvg[width=\linewidth]{III_Conclussion/im_time/train-iCharged.svg}
    %         \caption{Charged state}
    %     \end{subfigure}
    %     \hfill
    %     \begin{subfigure}[b]{0.475\textwidth}
    %         \centering
    %         \includesvg[width=\linewidth]{III_Conclussion/im_time/train-iDischarging.svg}
    %         \caption{Discharging process}
    %     \end{subfigure}
    %     \begin{subfigure}[b]{0.475\textwidth}
    %         \centering
    %         \includesvg[width=\linewidth]{III_Conclussion/im_time/train-iDischarged.svg}
    %         \caption{Discharged state}
    %     \end{subfigure}
    %     \caption{Different initial periods of model validation}
    %     \label{fig:init_time}
    % \end{figure}
    % Plots above outline different initial starting points and show the convergence over time. \textcolor{red}{Stupid idea, that's not helpful. How about I add some red line at very beginning, indicated where did I start initially and how far it plotted itself. Accumulate 500 initials steps and then plot them seperately on top to show the indication.}
    \begin{figure}[htbp]
        \centering
        \begin{subfigure}[b]{0.325\textwidth}
            \centering
            \includesvg[width=\linewidth]{III_Conclussion/Models/Sadykov2021-30steps/FUDS-models/SMRFUDS-FF-19.svg}
            \caption{FUDS trained model}
            \label{subfig:FUDS_diff_prof_compare}
        \end{subfigure}
        \hfill
        \begin{subfigure}[b]{0.325\textwidth}
            \centering
            \includesvg[width=\linewidth]{III_Conclussion/Models/Sadykov2021-30steps/FUDS-models/SMRFUDS-Test One-19.svg}
            \caption{Testing against DST profile}
            \label{subfig:DST_diff_prof_compare}
        \end{subfigure}
        \hfill
        \begin{subfigure}[b]{0.325\textwidth}
            \centering
            \includesvg[width=\linewidth]{III_Conclussion/Models/Sadykov2021-30steps/FUDS-models/SMRFUDS-Test Two-19.svg}
            \caption{Testing against US06 profile}
            \label{subfig:US_diff_prof_compare}
        \end{subfigure}
        \caption{Different initial periods of model validation}
        \label{fig:diff_prof_compare}
    \end{figure}
    % \begin{itemize}
    %     \item To verify the efficiency, model also was compared against the two other profiles cycling profiles. \\
    %     \item \textbf{model has tendency to put a lot of weight into the Voltage. This way. weights will fall into SoC instead to preserve longer dependency.} \\
    %     \item \textit{Different comparison, plots. Relative timing and what mechanism in implementation with lists and tensors increased it. Accuracies.} \\
    %     \item \textbf{Comaprison of different Windows size. Time it takes to compute that is too long. Need to overcome it, somehow.} \\
    % \end{itemize}
    \begin{figure}
        \centering
        \includesvg[width=\linewidth]{III_Conclussion/Models/Sadykov2021-30steps/FUDS-models/SMRFUDS-performance.svg}
        \caption{Accuracy evolution over training process for Mean Average Error (MAE) and Root Mean Squared Error (RMSE)}
        \label{fig:res_performance}
    \end{figure}
\subsection{Optimal output step size}
    The optimal size of the output steps for better capturing the State of Charge was determined through experimental training. Table~\ref{tab:out_steps} outlines the training iterations of each driving profile until it reached accuracy below 2.5\%.
    \begin{table}[htbp]
        \centering
        \caption{Percentage accuracy evaluation with increasing output step size}
        \label{tab:out_steps}
        \begin{tabular}{ p{1.5cm} || p{1.5cm} p{1.5cm} p{1.5cm} || p{1.5cm}  }
            \hline
            Steps & DST & US06 & FUDS & Epochs \\
            \hline
            10 & 2.94 & $\geq\sim 15$ & $\geq\sim 15$ & 8 \\
            15 & 0.90 & 10.67  & 11.10  & 8 \\
            20 & |    & 0.68   &  4.25  & 10 \\
            25 & |    & |      &  3.48  & 25 \\
            30 & |    & |      &  0.59  & 19 \\
            \hline
        \end{tabular}
    \end{table}
    
    %
    %
    The Dynamic Stress Test (DST) has a stable trend of current consumption and regenerative process.
    Therefore it did not take long to capture the behaviour of the charge.
    A single cycle on subfigure~\ref{subfig:res_DST} provides an example of accurate feed-forward charge estimation with output steps between 10-15 samples.
    The US06 and FUDS have the more aggressive current usage, leading to trendier voltage readings.
    As per subfigures~\ref{subfig:res_US} and~\ref{subfig:res_FUDS}, it took twice more output samples to achieve similar accuracy to DST based model.
    Any further increase can capture even more complicated scenarios, leading to a longer training time.
    However, the epoch per training model to produce the best fit decreased with every added step.
    
    %
    %
    An optimal step is required for running the training process on the entire dataset with all three profiles as inputs to prepare the deployment model.
    Therefore, all other results demonstrations will use 30 steps as the most optimal and report any other results based on that parameter.
    \begin{figure}[htbp]
        \centering
        % DST based tests
        \begin{subfigure}[b]{0.325\textwidth}
            \centering
            \includesvg[width=\linewidth]{III_Conclussion/Models/Sadykov2021-15steps/DST-models/SMRDST-FF-8.svg}
            \caption{Best DST based model after 8 iterations with 15 steps}
            \label{subfig:res_DST}
        \end{subfigure}
        \hfill
        \begin{subfigure}[b]{0.325\textwidth}
            \centering
            \includesvg[width=\linewidth]{III_Conclussion/Models/Sadykov2021-20steps/US06-models/SMRUS06-FF-10.svg}
            \caption{Best US06 based model after 10 iterations with 20 steps}
            \label{subfig:res_US}
        \end{subfigure}
        \hfill
        \begin{subfigure}[b]{0.325\textwidth}
            \centering
            % \includesvg[width=\linewidth]{III_Conclussion/im_best/FUDS-19.svg}
            \includesvg[width=\linewidth]{III_Conclussion/Models/Sadykov2021-30steps/FUDS-models/SMRFUDS-FF-19.svg}
            \caption{Best FUDS based model after 19 iterations with 30 steps}
            \label{subfig:res_FUDS}
        \end{subfigure}
        \caption{Best training results over 3 different driving profiles}
        \label{fig:Models_res}
    \end{figure}

\section{Conclusion} \label{sec:conclussion}
\section{Conclusion} \label{sec:conclussion}
What was the goal, what was achieved and what will be applied for On-board computer 

\section*{Acknowledgments}
% The authors would like to thank...
\section{Acknowledgements} \label{sec:acknowledgements}
The research was undertaken through funding from the Autonomous...
\textcolor{red}{Following commented loop needs to be replaced with single huge fugire demonstrating performance evolution relative to other profiles}
% \foreach \i in {1,2,...,20}{
%     \begin{figure}[htbp]
%         \centering
%         \includesvg[width=\linewidth]{III_Conclussion/images/\i.svg}
%         \caption{Epoch/Iteration \i}
%     \end{figure}
% }


\printcredits

% \funding{
%     The research was funded by the Automotive Engineering Graduate Program (AEGP), grand number AEGP000036, from the Australian Government Department of Industry Science, Energy and Resources in cooperation with industry partner Prohelion.
% }

% \dataavailability{
%     \textls[-5]{The lithium-ion battery cycling data presented in this study are openly available at the Battery Research Group of the Center for Advanced Life Cycle Engineering (CALCE) Group official repository website, which can be accessed using the following link: \url{https://web.calce.umd.edu/batteries/data.htm}, in the section on A123 batteries~\cite{noauthor_calce_2017} (accessed on 20 March~2020).}
% }

% \conflictsofinterest{
%     The funders had no role in the design of the study; in the collection, analyses, or interpretation of data; in the writing of the manuscript; or in the decision to publish the results.
% } 

%% Loading bibliography style file
%\bibliographystyle{model1-num-names}
\bibliographystyle{cas-model2-names}

% Loading bibliography database
\bibliography{
  ../Ch3Analysis/I_Introduction/BibIntro,
  ../Ch3Analysis/II_Body/BibRNN,
  ../Ch3Analysis/II_Body/GRU/BibGRU,
  ../Ch3Analysis/II_Body/LSTM/BibLSTM,
  I_Introduction/BibIntro,
  ../Z-References/References
}

\end{document}


