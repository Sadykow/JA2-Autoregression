%
%
\ifthenelse{\boolean{thesis}}{To carry out detailed evaluation of the new model, the battery cycling data obtained from the Battery Research Group of the Center for Advanced Life Cycle Engineering (CALCE) Group at the University of Maryland~\cite{noauthor_calce_2017}, as per methodology of Chapter~\ref{cha:Analysis}.}
{The battery cycling data used here were obtained from the Battery Research Group of the Center for Advanced Life Cycle Engineering (CALCE) Group at the University of Maryland~\cite{noauthor_calce_2017}.}
The data includes battery temperatures of 20, 25, 30, 40 and 50\textdegree{}C over three driving profiles and these have been selected as training and testing groups.
Each temperature was taken from three driving profiles: Dynamic Stress Test (DST), Highway driving (US06) and Federal-Urban driving schedule (FUDS) (see~\cite{castillo_18_2015}).
% The FUDS driving scheduler acts as a training set, then the other two DST and US06 for testing.
% The choice of validation sets was due to differences in the current consumption.
% If DST has a constant variation of the current amps over time, the US06 uses an aggressive mechanism similar to the FUDS.
The ability to accurately capture other driving mechanisms with minor error deviation from training, acts as a validation mechanism and justifies the value of the technique.
The charge cycles were resampled a discharge rate of 1 Hz, and all data was normalised based on a mean and standard deviation of the training set.
Figure~\ref{fig:cross-data} outlines the State of Charge cycles, associated with Voltage, Current and Temperature, oversampled and cross-cycle classified for Training, Validation and Testing purposes.
%!!!
%!!
\begin{figure}[ht]
    \centering
    \includesvg[width=\columnwidth]{../Ch3Analysis/II_Body/images/cross-data.svg}
    \caption{Three profiles cross data split for training, validation and testing in a simplistic SoC cycle representation through different temperatures. The three cycle types were DST, US06 and FUDS, with one selected as the training cycle, and the other two as testing for each evaluation.}
    \label{fig:cross-data}
\end{figure}

%
%
\ifthenelse{\boolean{thesis}}{The accuracy calculation is similar to what Chapter~\ref{cha:Analysis}, Subsection~\ref{subsec:t_model} and Figure~\ref{fig:plot_demo} demonstrated.
% The difference is that dotted prediction lines are distinguished by red and yellow colours, where red indicates a typical prediction at any time, and yellow is the continuous feed-forward where only an initial 500 were provided.
} {
\begin{figure}[ht]
    % RMSE equation: RMS = (tf.keras.backend.sqrt(tf.keras.backend.square(y_test_one[::,]-PRED)))
    \centering
    \includesvg[width=\columnwidth]{II_Body/images/plot-example.svg}
    \caption{Feed-Forward accuracy plot demonstration.}
    \label{fig:plot_demo}
\end{figure}
Figure~\ref{fig:plot_demo} shows an example of how prediction accuracy has been evaluated in this work, where the actual State of Charge is compared with prediction.
The filled area below the plot captures the error in terms of the Absolute Difference between two lines, as per Equation~\ref{eq:abs-error}.
% For comparison, all Figures with two subplots were distinguished from each other by prediction line color, as per early Figure~\ref{fig:regular_tr}.
% The left subfigures~\textit{a} with red output line referes to the input taken directrly from the table, including the perfect SoC history.
% The right subfigures~\textit{b} with yellow prediction refers to Feed-Forward method, where only Voltage, Current and Temperature were taken from the table through the testing.
Only the first input window during training/testing was considered ideal by the initial fully charged/discharged state or estimated with other NN methods.
The follow-up predictions are based on the early produced output of the same model in a feed-forward manner.
% No $R^2$ has been produced on the Feed-forward prediction, since it is mathematically not feasible at current state due to the mix of multiple outputs.
\begin{equation}
    \textbf{ABS error}  = \sqrt{(Actual-Prediction)^2}
    \label{eq:abs-error}
\end{equation}
}
\ifthenelse{\boolean{thesis}}{Overall, the result reporting is conducted the same way as it was in Chapter~\ref{cha:Analysis} on the average of 10 attempts with cross-testing on three profiles.}
{The stochastic nature of NN algorithm has been addressed by training each model 10 times, and using the average of the corresponding 10 predictions as the actual.
In this way, outlying models and random fluctuations are smoothed to provide a much better and more accurate representation following the standard ML process (see~\cite{sadykov_practical_2022} for more details).}
% Report resutls based on error degradation pver Mean Average Error and Root MEan Squared Errors.
% Choosing the most optimal will be the test subject.
\ifthenelse{\boolean{thesis}}{In the end, to compute overall performance comparable results, each model was tested against the entire training set of all three profiles and summarised in a single table, compared with two LSTM models from previous Chapter~\ref{cha:Analysis}.}
{Overall, the results were reported on nine subplots, three per profile, where the first column would outline training and testing history over training time, the second - predictions on a single 25\textdegree{}C trained cycle, and the third from 25 and 30\textdegree{}C testing against cycles of other two profiles.
% In the end, each model was tested against the entire training set of all three profiles and summarised in a single table, compared with two LSTM models from early published article~\cite{sadykov_practical_2022}. %? RETURN THIS if space would allow it
}
% BEst one per each iteration will be summarised in a table and then compared against two acccuracy plots as per following example.