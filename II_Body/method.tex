%
%
The battery cycling data were obtained from the Battery Research Group of the
Center for Advanced Life Cycle Engineering (CALCE) Group
at the University of Maryland~\cite{noauthor_calce_2017}.
The temperatures of 20, 25, 30, 40 and 50\textdegree{}C over three driving profiles were selected as training and testing groups.
Each temperature was taken from three driving profiles: Dynamic Stress Test (DST), Highway driving (US06) and Federal-Urban driving schedule (FUDS)
% The FUDS driving scheduler acts as a training set, then the other two DST and US06 for testing.
% The choice of validation sets was due to differences in the current consumption.
% If DST has a constant variation of the current amps over time, the US06 uses an aggressive mechanism similar to the FUDS.
The ability to accurately capture other driving mechanisms with minor error deviation from training acts as a validation mechanism and justifies the value of the technique.
The charge cycles were resampled to equalise with a discharge rate of 1 Hz, and all data were normalised based on a mean and standard deviation of the training set.

%
%
\ifthenelse{\boolean{thesis}}{
The accuracy calculation is similar to what Chapter~\ref{cha:Analysis}, Subsection~\ref{subsec:t_model} and Figure~\ref{fig:plot_demo} demonstrated.
The difference is that dotted prediction lines are distinguished by red and yellow colours, where red indicates a typical prediction at any time, and yellow is the continuous feed-forward where only an initial 500 were provided.
} {
\begin{figure}[ht]
    % RMSE equation: RMS = (tf.keras.backend.sqrt(tf.keras.backend.square(y_test_one[::,]-PRED)))
    \centering
    \includesvg[width=0.8\columnwidth]{II_Body/images/plot-example.svg}
    \caption{Feed-Forward accuracy plot demonstration.}
    \label{fig:plot_demo}
\end{figure}
Figure~\ref{fig:plot_demo} shows an example of accuracy evaluation, where the actual State of Charge is compared with prediction.
The filled area below the plot captures the error Absolute Error Difference between two lines, as per Equation~\ref{eq:abs-error}.
% For comparison, all Figures with two subplots were distinguished from each other by prediction line color, as per early Figure~\ref{fig:regular_tr}.
% The left subfigures~\textit{a} with red output line referes to the input taken directrly from the table, including the perfect SoC history.
% The right subfigures~\textit{b} with yellow prediction refers to Feed-Forward method, where only Voltage, Current and Temperature were taken from the table through the testing.
Only the first sample window was taken from other estimation methods and considered ideal, either by the initial fully charged or discharged state or estimated with other simpler NN methods.
Everything follow-up prediction is based on the early produced output of the same model.
% No $R^2$ has been produced on the Feed-forward prediction, since it is mathematically not feasible at current state due to the mix of multiple outputs.
\begin{equation}
    \textbf{ABS error}  = \sqrt{(Actual-Prediction)^2}
    \label{eq:abs-error}
\end{equation}
}
Overall, the results were reported on six average of ten attempts subplots, three per profile, where the first would outline training and testing history over training time and the other two predictions on a single 25\textdegree{}C trained cycle, and two from 25 and 30\textdegree{}C testing cycles of other two profiles.
% Report resutls based on error degradation pver Mean Average Error and Root MEan Squared Errors.
% Choosing the most optimal will be the test subject.
In the end, to compute overall performance comparable results, each model was tested against the entire training set of all three profiles and summarised in a single table.
% BEst one per each iteration will be summarised in a table and then compared against two acccuracy plots as per following example.