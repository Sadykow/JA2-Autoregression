The battery cycling data were obtained from the Battery Research Group of the
Center for Advanced Life Cycle Engineering (CALCE) Group
at the University of Maryland~\cite{noauthor_calce_2017}.
The temperatures of 20,25,30 degrees over three driving profiles were selected as training and testing groups.
Each temperature was taken from three driving profiles: Dynamic Stress Test (DST), Highway driving US06 and Federal-Urban driving schedule (FUDS)
The FUDS driving scheduler acts as a training set, then the other two DST and US06 for testing.
The choice of validation sets was due to differences in the current consumption.
If DST has a constant variation of the current amps over time, the US06 uses an aggressive mechanism, similar to the FUDS.
The ability to accurately capture both driving mechanisms with minor error acts as a validation mechanism and justifies the value of the technique.

%
%
Report resutls based on error degradation pver Mean Average Error and Root MEan Squared Errors.
Choosing the most optimal will be the test subject.

%
%
BEst one per each iteration will be summarised in a table and then compared against two acccuracy plots as per following example.

%
%
Figure~\ref{fig:plot_demo} shows an example of accuracy evaluation, where actual State of Charge compared with prediction.
The filled area below the plot captures the error Absolute Error Difference between two lines, as per the Equation~\ref{eq:abs-error}.
For the purpose of comparioson, all Figures with two subplots were distiguished from each other by prediction line color, as per early Figure~\ref{fig:regular_tr}.
The left subfigures~\textit{a} with red output line referes to the input taken directrly from the table, including the perfect SoC history.
The right subfigures~\textit{b} with yellow prediction refers to Feed-Forward method, where only Voltage, Current and Temperature were taken from the table through the testing.
Only the first windows of samples were taken from other sources and considered ideal, either due to be the initial fully charge or discharge state, or estimated with other methods.
Everything following prediction is based on the early produced output of the same model.
No $R^2$ has been produced on the Feed-forward prediction, since it is mathematically not feasible at current state due to the mix of multiple outputs.
\begin{figure}[ht]
    % RMSE equation: RMS = (tf.keras.backend.sqrt(tf.keras.backend.square(y_test_one[::,]-PRED)))
    \centering
    \includesvg[width=\columnwidth]{II_Body/images/plot-example.svg}
    \caption{Accuracy plot demonstration.}
    \label{fig:plot_demo}
\end{figure}
\begin{equation}
    \textbf{ABS error}  = \sqrt{(Actual-Prediction)^2}
    \label{eq:abs-error}
\end{equation}