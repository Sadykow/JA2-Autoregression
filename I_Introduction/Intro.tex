%
% Intro to chapter with Li-ion batteries
\ifthenelse {\boolean{thesis}}
{
    The rapid growth of the Electrical Vehicle (EV) market has led to the research of better energy storage.
} {
    \IEEEPARstart{T}{he} rapid growth of the Electrical Vehicle (EV) market has led to the research of better energy storage.
}
Lithium-Ion (Li-ion) batteries offer high power density and a low self-discharge rate~\cite{han_review, en13082106}.
Such efficiency made them highly used in the EV application, on top of smartphones or other portable electronic devices.
Misusing those batteries may lead to rapid battery life degradation (ageing) and damage to a device on a small scale or environmental catastrophes on a big.
The only solution to an inefficiently used aged battery is recycling and replacement, increasing the usage cost~\cite{skeete_beyond_2020}.
The vital requirement for EVs powered with Li-ion batteries is the safe and efficient operation of the integrated Battery Management System (BMS).

%
% SoC estimation current state
State of Charge became an issue of great importance in the BMS application.
Since one of the properties of a manufactured Li-ion battery is internal resistance, the voltage reading over a cell becomes unreliable during unstable discharge.
The voltage drop leads to inaccurate voltage and temperature sensor readings during a system operation.
Since the depletion of a battery until it cannot supply enough current is undesirable for more prolonged usages, it is crucial to implement a solution to overcome such issues.
A simple battery model that accumulates the multiple of internal resistance and current consumed by a system became a commonly used solution.
Due to battery ageing over more prolonged use, the resistance diversifies due to varying battery utilised temperatures.
Eventually, such a simple battery model becomes inefficient and only leads to failure~\cite{fenrg.2019.00065}.
Working temperature, stored charge and maximum output current differed from a freshly manufactured cell.
Therefore, failure to prevent overrated usage leads to battery damage and wasting cell potential earlier than anticipated by manufacturers.
The solution is to use the State of Charge (SoC) as a battery state reference.

%
% Other SoC methods
Several methods have been used to make an accurate estimation.
The most commonly used and simple to implement is the Coulomb Counter (CC), integral of the current over time.
However, besides the limitation of not telling the initial SoC, it cannot converge to the actual State of Charge over more prolonged usages~\cite{ng_enhanced_2009}.
The reason for that is the method itself.
The CC model does not consider variations in the battery properties due to degradation, reporting charge percentage by a scale from the first usage.
% \textcolor{red}{(Several other studies if I want to reference them).}
Different and more complicated methods approach this problem in a chemical or complex modelling way.
The methods explored in this work are based on statistical Neural Network models.

%
% A state of RNN at SoC
In recent years, there have been a few different approaches in Artificial Intelligence development for battery charge estimation: Fuzzy logic~\cite{malkhandi_fuzzy_2006}, Support Vector Machine~\cite{hansen_support_2005, anton_battery_2013}, and Recurrent Neural Network (RNN)~\cite{song_lithium-ion_2018,Chemali2017,mamo_long_2020,jiao_gru-rnn_2020,xiao_accurate_2019,javid_adaptive_2020,zhang_deep_2020}.
The RNNs are more commonly used due to their various applications, support from different developers, and being widely tested by researchers.
There have been multiple publications around extensively used Time-Series models, such as Long short-term memory (LSTM) or Gated recurrent units (GRU), for battery state of charge prediction.
\ifthenelse {\boolean{thesis}}
{
    The Chapter~\ref{cha:Analysis} has analysed and discussed multiple implementations of RNN models to determine the State of Charge from sensory data like Voltage, Current and Temperature.
    The research results have concluded that sensory data in one driving scenario effectively predicts that specific driving behaviour but is poor at extrapolating to others by doubled error at best.
    Thus, the battery state became a matter of not the battery itself but the accuracy and usage of the measurements.
    Besides, the internal resistance of batteries creates voltage drops during the current drain.
    The ability to accommodate such loss and still keep track of the State of Charge defines the quality of the ML model.
    The potential solution to the problem was integrating SoC as one of the input features to a NN model.
    The limitation here is to get the State of Charge in real-time.
    There are two approaches to that: Charge estimation from other means, CC or three feature-based NN methods, or the Feed-Forward method to propagate the output value of the charge as an input into the next prediction.
} {
    \textcolor{red}{A previously published paper of this research [Sadykov, 1] analysed multiple different implementations of RNN models to determine the State of Charge from sensory data like Voltage, Current and Temperature.}
    The research results have concluded that sensory data in one driving scenario effectively predicts that specific driving behaviour but is poor at extrapolating to others by doubled error at best.
    Thus, the battery state became a matter of not the battery itself but the accuracy and usage of the measurements.
    Besides, the SoC can fall or rise below/high under/over-voltage limit under instantaneous high current spikes, although not as significant to trigger an alarm and only at the end of discharge/charge cycles due to batteries' ageing internal resistance.
    The potential solution to the problem was integrating SoC as one of the input features to a NN model.
    The limitation here is to get the State of Charge in real-time.
    There are two approaches to that: Charge estimation from other means, CC or three feature-based NN methods, or the Feed-Forward method to propagate the output value of the charge as an input into the next prediction.
}


%
% Papers purpose
\ifthenelse {\boolean{thesis}}
{
    The correction of the limitations identified in Chapter~\ref{cha:Analysis}, such as a training cross-accurate model and minimisation of resistance influence, will require implementing a new 4-feature-based method.
    In this chapter, the Long short-term memory model will be developed, with a history of approximately 8 minutes and 20 seconds usage (500 samples at 1Hz), to predict the current State of Charge and propagate the result further into the next prediction.
    A novel method of the training loop model will be proposed to eliminate the possibility of accumulating error along with the value of a charge, utilising the Autoregressive technique to introduce an adaptive and robust solution to training output inaccuracy over time.
    % It utilises the Autoregressive technique to introduce an adaptive and robust solution to training with some training output inaccuracy over time.
    Furthermore, it forces the model to consider the potential of having variations in the State of Charge history and not fail the prediction in the following outputs.
} {
    In this paper, the Long short-term memory model will implement a 4-feature estimator, with a history of 8 minutes and 20 seconds usage (500 samples at 1Hz), to predict the current State of Charge and propagate the result further.
    A novel method of the training loop model will be proposed to eliminate the possibility of accumulating error along with the value of a charge, utilising the Autoregressive technique to introduce an adaptive and robust solution to training output inaccuracy over time.
    Furthermore, the model is forced to consider a miss accuracy of its State of Charge prediction before loss calculation.
}

%
%
\ifthenelse {\boolean{thesis}}
{
    This chapter is organised as follows: a methodology for an RNN model discussed in Section~\ref{sec:layer}.
    The details of how auto-regression has been utilised are in Section~\ref{sec:feed}.
    Section~\ref{sec:AU:Results} summarises the investigation results in a comparable manner similar to early mentioned work and compares the degree of improvement over traditional RNN methods.
    Finally, section~\ref{sec:AU:Conclussion} concludes the research by outlining several observations, which may require separate consideration.
} {
    This paper is organised as follows: a methodology for an RNN model discussed in Section~\ref{sec:layer}.
    The details of how auto-regression has been utilised are in Section~\ref{sec:feed}.
    % Subsections 4.1 and 4.2 separate model validation points and parameter estimation processes.
    Section~\ref{sec:results} summarises the investigation results in a comparable manner similar to early mentioned work and compares the degree of improvement over traditional RNN methods.
    Finally, section~\ref{sec:conclussion} concludes the research by outlining several observations, which may require separate consideration.
}
% Most were isolated to closed scenarios with provided data or from battery cycling machines.
% The most promising approach to improve a model and make it more universal is to increase complexity. While some introduced deeper layer newtwork, others added additional mechanisms to already used.
