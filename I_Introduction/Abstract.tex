%
% IF
\ifthenelse {\boolean{thesis}} {
Implementing a reliable algorithm for determining the State of Charge (SoC) of a battery in an electrical grid is one of the significant tasks in the Battery Management System (BMS), such as Lithium-Ion accumulators in an Electric Vehicle.
There have been numerous methods to simulate battery behaviour to calculate the exact charge percentage and Remaining Useful Life (RUL).
The Machine Learning (ML) battery models use statistical data of battery utilisation over long-term usage to predict SoC or estimate the number of remaining charge cycles.
The most common 3-feature models are based on sensory data, such as Voltage Current and Temperature.
In the present study, a novel approach of charge estimation has been proposed using the Feed-Forward technique with 4-feature Autoregressive Long Short-Term Memory Recurrent Neural Network (LSTM-RNN) model.
The outputted SoC value becomes the 4th feature in the input matrix, along with Voltage, Current and Temperature.
The usual RNN models tend to solve the vanishing gradient problem, but the Autoregressive modification of a training loop allows to decompress the training into individual time steps and fed each output back to the model.
The technique is intended to accommodate the error in the possible prediction so that the feed-forward approach will not accumulate miss accuracy with every following output.
The proposed technique is implemented on current Lithium-Ion battery cycling data and validated using the cross-validation technique on three different current profiles: Dynamic Stress Test (DST), Supplemental Federal Test Procedure-driving Schedule (US06) and Federal Urban Driving Schedule (FUDS).
Unlike simple Feed-Forward prediction with high error due to accumulated  SoC offset, the Autoregressive modification lowered the training miss accuracy to 1-3\% of Mean Average Error (MAE).
The implementation has been modified to fit low-power Chip-on-Board devices or used on TPU processors to be easily integrated into any electrical grid.
}{
Implementing a reliable algorithm for determining the State of Charge (SoC) of a battery in an electrical grid is one of the significant tasks in the Battery Management System (BMS).
Accurate and universal SoC estimation at different utilisation conditions helps assess the batteries' performance and improve overall usage.
Given the variety of issues with traditional methods of calculating SoC, the Computer Intelligence-based models tend to use statistical data rather than battery modelling, which allows charge estimation from an unknown initial charge percentage at any point of utilisation.
Due to the common non-linear characteristics of Lithium-Ion cells, the Neural Network based models have had better performance results in building the complicated multidimensional relationship between typical battery sensory data.
%: Voltage, Current and Temperature, yet fail to allocate weight across trendies features by themselves.
However, using pure history of sensory measurements of Voltage, Current and Temperature over a full cycle battery utilisation has imperfections when applying trained models on unforeseen driving tests.
In the present paper, a Recurrent Neural Network (RNN) model with a new Autoregressive modification has been proposed to improve upon early published Machine Learning methods to enhance the estimation accuracy further and make results less sensible to changed conditions of usage.
This method has an advantage over traditional 3-feature-based models to be used in the Feed-forward way to preserve SoC results for further estimation and minimise accumulated error influence on the overall SoC prediction system.
As a result, the method could diverge with actual trained State of Charge values from any initial condition and accurately capture even the most complicated charge regions within 2-3\% error on the testing datasets.
% (?) It also difficult to determine SoC at a unknows initial conditions, which makes NN superior method... 
% Has a potential of predicting in future on how much remaining usefull charge can be used
}