%
% IF
\ifthenelse {\boolean{thesis}}
%
% THEN
{
Implementing a reliable algorithm for determining the State of Charge (SoC) of a battery in an electrical grid is one of the significant tasks in the Battery Management System (BMS), such as Lithium-Ion accumulators in an Electric Vehicle.
There have been numerous methods to simulate battery behaviour to calculate the exact charge percentage and Remaining Useful Life (RUL).
The Machine Learning (ML) battery models use statistical data of battery utilisation over long-term usage to predict SoC or estimate the number of remaining charge cycles.
The most common 3-feature models are based on sensory data, such as Voltage Current and Temperature.
In the present study, a novel approach of charge estimation has been proposed using the Feed-Forward technique with 4-feature Autoregressive Long Short-Term Memory Recurrent Neural Network (LSTM-RNN) model.
The outputted SoC value becomes the 4th feature in the input matrix, along with Voltage, Current and Temperature.
The usual RNN models tend to solve the vanishing gradient problem, but the Autoregressive modification of a training loop allows to decompress the training into individual time steps and fed each output back to the model.
The technique is intended to accommodate the error in the possible prediction so that the feed-forward approach will not accumulate miss accuracy with every following output.
The proposed technique is implemented on current Lithium-Ion battery cycling data and validated using the cross-validation technique on three different current profiles: Dynamic Stress Test (DST), Supplemental Federal Test Procedure-driving Schedule (US06) and Federal Urban Driving Schedule (FUDS).
Unlike simple Feed-Forward prediction with high error due to accumulated  SoC offset, the Autoregressive modification lowered the training miss accuracy to 1-3\% of Mean Average Error (MAE).
The implementation has been modified to fit low-power Chip-on-Board devices or used on TPU processors to be easily integrated into any electrical grid.
}
%
% ELSE
{
 ... \\
 (6) In the present paper, a RNN model with a new Autoregressive modification is proposed to improve upon earlt published ML-based methods.
 (7) This method has advange over tradional 3-feature based models to be used in the Feed-forward form/technique to preserve SoC results for further estimation and minimise previos error influence.
}