%
% IF
\ifthenelse {\boolean{thesis}} {
Implementing a reliable algorithm for determining the State of Charge (SoC) of a battery in an electrical grid is one of the significant tasks in the Battery Management System (BMS), such as Lithium-Ion accumulators in an Electric Vehicle.
There have been numerous methods to simulate battery behaviour to calculate the exact charge percentage and Remaining Useful Life (RUL).
The Machine Learning (ML) battery models use statistical data of battery utilisation over long-term usage to predict SoC or estimate the number of remaining charge cycles.
The most common 3-feature models are based on sensory data, such as Voltage Current and Temperature.
In the present study, a novel approach of charge estimation has been proposed using the Feed-Forward technique with 4-feature Autoregressive Long Short-Term Memory Recurrent Neural Network (LSTM-RNN) model.
The outputted SoC value becomes the 4th feature in the input matrix, along with Voltage, Current and Temperature.
The usual RNN models tend to solve the vanishing gradient problem, but the Autoregressive modification of a training loop allows to decompress the training into individual time steps and fed each output back to the model.
The technique is intended to accommodate the error in the possible prediction so that the feed-forward approach will not accumulate miss accuracy with every following output.
The proposed technique is implemented on current Lithium-Ion battery cycling data and validated using the cross-validation technique on three different current profiles: Dynamic Stress Test (DST), Supplemental Federal Test Procedure-driving Schedule (US06) and Federal Urban Driving Schedule (FUDS).
Unlike simple Feed-Forward prediction with high error due to accumulated  SoC offset, the Autoregressive modification lowered the training miss accuracy to 1-3\% of Mean Average Error (MAE).
The implementation has been modified to fit low-power Chip-on-Board devices or used on TPU processors to be easily integrated into any electrical grid.
}{
    % \begin{itemize}
        % \item This paper presents \dots
        This paper presents a Recurrent Neural Network (RNN) model for the State of Charge (SoC) estimation of Electric Vehicle.
        A new Autoregressive modification is proposed to improve upon earlier published Machine Learning models to enhance the prediction accuracy and make results less sensitive to interchangeable usage conditions.
        % \item Background \dots How?
        By building upon earlier published findings, assessing several models ability to estimate SoC using a history window of sensory data (Sadykov \textit{et al}~\cite{sadykov_practical_2022}), this work proposes, implements, and reports its methods of improving accuracy by introducing the State of Charge as part of sensory inputs in a Feed-Forward manner.
        % \item Method \dots -> What have been done
        The method is based on the Autoregressive implementation of RNN model training, where successively calculated SoC values are used as input in the subsequent predictions, but the potential inaccuracies with those predictions that might otherwise cause significant and rapid divergence are accounted for within a modified recursive training procedure.
        %the several latest samples of known SoC in the inputs are considered potentially inaccurate and treated as a source of divergence in the follow-up predictions during Feed-Forward testing that optimisation process is meant to minimise.
        % the main issue for model divergence during Feed-Forward prediction and made to be one of the features
        % \item Results \dots What results where achieved.
        As a result, with a slight increase in training time, the accuracy of the output prediction for three different realistic charge-discharge driving profiles doubled compared to the two best previously published RNN models.
        % \item conclusions \dots What conclussion have I wrote?
        Overall, the test results have demonstrated the usability of the current model in actual driving scenarios, making such models a viable replacement to estimation approaches used in electric vehicle battery management systems.
        %which in combination with any other RNN methods to determine the initial set of input State of Charge and theoretically low computational load on Tensor processors, can become a suitable replacement to any other traditional way of estimation, which relies on battery modelling rather than statistical estimations.
    % \end{itemize}
% Implementing a reliable algorithm for determining the State of Charge (SoC) of a battery in an electrical grid is one of the significant tasks in the Battery Management System (BMS).
% Accurate and universal SoC estimation at different utilisation conditions helps assess the batteries' performance and improve overall usage.
% Given the variety of issues with traditional methods of calculating SoC, the Computer Intelligence-based models tend to use statistical data rather than battery modelling, which allows charge estimation from an unknown initial charge percentage at any point of utilisation.
% Due to the common non-linear characteristics of Lithium-Ion cells, the Neural Network based models have had better performance results in building the complicated multidimensional relationship between typical battery sensory data.
% %: Voltage, Current and Temperature, yet fail to allocate weight across trendies features by themselves.
% However, using pure history of sensory measurements of Voltage, Current and Temperature over a full cycle battery utilisation has imperfections when applying trained models on unforeseen driving tests.
% In the present paper, 
% This method has an advantage over traditional 3-feature-based models to be used in the Feed-forward way to preserve SoC results for further estimation and minimise accumulated error influence on the overall SoC prediction system.
% As a result, the method could diverge with actual trained State of Charge values from any initial condition and accurately capture even the most complicated charge regions within 2-3\% error on the testing datasets.
% (?) It also difficult to determine SoC at a unknows initial conditions, which makes NN superior method... 
% Has a potential of predicting in future on how much remaining usefull charge can be used
}